\subsection{Απάντηση Υποερωτήματος (β)}
\label{ssec:Solution_2.2}
\doublespacing
Έχουμε να κατασκευάσουμε PDA για CFL $L_1 = \{b^n a^k b^k a^n : k, n \in  \mathbb{N}^0 \}$\\
Από τον τύπο δυνάμεθα να εξάγουμε τα εξής δεδομένα, για συμβολοσειρά $w \in L_1$:

\begin{itemize}
	\itemsep0em
	\item $\vert w \vert_{min} = 0 \Longrightarrow n,\,k = 0 \Longrightarrow w = ε$, \quad $\vert w \vert_{max} =
	\aleph_0$, \quad $|w| = 2(n+k) = \text{Άρτιο}$
	\item $ n + k - 2nk = 1  \Longrightarrow |w|_{min} = 2 \Longrightarrow w = ba \cup ab$
	\item $ n, k = 1 \Longrightarrow \vert w \vert_{min} = 4 \Longrightarrow w = baba$, \qquad $L \subseteq \{xx^R
	\vert x \in \Sigma^* \}$
	\item Για μεγαλύτερα $n, k$ έχουμε συμβολοσειρές όπως $bbabaa, baabba, bbaabbaa...$
\end{itemize}

\par
Από αυτά κάποια δεν μας λένε κάτι ιδιαίτερα χρήσιμο, άλλα όμως μας φανερώνουν την οδό που θα πρέπει να
ακολουθήσουμε. Για παράδειγμα, κατά αντιστοιχία με τις αντίστοιχες τεχνικές που ακολουθήσαμε στις
λύσεις ασκήσεων κανονικών εκφράσεων, τα διάφορα ελάχιστα μας δίνουν την βάση πάνω στην οποία θα χτιστεί το (PD)
αυτόματο.
\par
Ως εκ τούτου έχουμε φαινομενικά τέσσερεις ελάχιστες περιπτώσεις, που η
συγκεκριμένη γλώσσα επιτρέπει:
$w_0 = ε \lor w_1 = ab \lor w_2 = ba \lor w_3 = baba$ αλλά προσοχή, το $w = baba$
αποτελεί παρεμβολή του $w_1$ εντός του $w_2$ και όχι $(ba)^*$
\par
Αυτό με τη σειρά του δείχνει ότι ουσιαστικά, το $w_3$ αποτελεί επέκταση της $w_2$ μετά αρχικού $b$ και άρα δεν
είναι μέρος της καθολικής βάσης του αυτομάτου. Θα μας χρειαστεί όμως, αφού δείχνει ότι όταν ο
αλγόριθμος θα μπαίνει στη διαδρομή που διαβάζει $ba$, θα διακλαδώνεται σε δύο εναλλακτικές βάση του
αν αυτό που έπεται του $b$ είναι $a$ ή $aba$.
\vfill
\clearpage
\par Τέλος αφού $n, k \in \mathbb{N_0}$ τότε μπορούν όλοι αυτοί οι χαρακτήρες να είναι (μετρήσιμα) άπειροι αλλά υπό
τον όρο ότι τηρείται το γενικό μοτίβο.
\par Άρα, υποθετικά, έχουμε τρεις αρχικούς κλάδους και σε έναν από αυτούς μία διάσπαση σε δύο. Θα δούμε ότι ακριβώς
αυτό συμβαίνει και στην πράξη (τουλάχιστον στη συγκεκριμένη επίλυση) και ότι γενικά είναι καλή
πρακτική. Ξεκινάμε κατασκευή αυτομάτου, αρχικά δίνοντας μαθηματική περιγραφή και κατόπιν αντίστοιχο
διάγραμμα:

\begin{tcolorbox}[colback=yellow!15!white, colframe=blue!50!white,
	fonttitle=\bfseries\Large, title = {PDA $M = (K,\, \Sigma,\, \Gamma,\, \Delta,\, s,\, F)$}]
	\begin{itemize}
		\itemsep0em
		\item $K \,=\, \{q1,\, q2,\, q3,\, q4,\, q5,\, q6\}$
		\item $\Sigma \,=\, \{a,\, b\}$
		\item $\Gamma \,=\, \{A,\, B,\, \$\}$
		\item $\Delta \,=\,$
		\reducevspace\reducevspace\reducevspace\reducevspace\reducevspace\reducevspace\reducevspace\reducevspace
		\begin{multicols}{2}
			\begin{enumerate}
				\item $(q1,\,ε,\,ε)\rightarrow(q2,\,\$)$



				\item $(q2,\,b,\,\$)\rightarrow(q2,\,B\$)$
				\item $(q2,\,b,\,B)\rightarrow(q2,\,BB)$

				\item $(q2,\,ε,\,\$)\rightarrow(q6,\,ε)$
				\item $(q2,\,a,\,\$)\rightarrow(q3,\,A\$)$
				\item $(q2,\,a,\,B)\rightarrow(q3,\,AB)$
				\item $(q2,\,a,\,B)\rightarrow(q5,\,ε)$



				\item $(q3,\,a,\,A)\rightarrow(q3,\,AA)$

				\item $(q3,\,b,\,A)\rightarrow(q4,\,ε)$



				\item $(q4,\,b,\,A)\rightarrow(q1,\,ε)$

				\item $(q4,\,a,\,B)\rightarrow(q5,\,ε)$
				\item $(q4,\,ε,\,\$)\rightarrow(q6,\,ε)$



				\item $(q5,\,a,\,B)\rightarrow(q5,\,ε)$

				\item $(q5,\,ε,\,\$)\rightarrow(q6,\,ε)$
			\end{enumerate}
		\end{multicols}
		\reducevspace\reducevspace\reducevspace\reducevspace
		\item $s \,=\, q_1$
		\item $F \,=\, q_6$
	\end{itemize}
\end{tcolorbox}

%\vfill
%\clearpage
\begin{tcolorbox}[colback=yellow!15!white, colframe=blue!50!white,
	fonttitle=\bfseries\Large, title = DFA $M_{\text{τελικό}}$]
	\centering
	\begin{tikzpicture} [blue, node distance = 2.5cm, on grid, font=\sffamily\large\bfseries]
		% Help grid
		%	\draw [help lines] (-1,4) grid (8,-10);
		% Start Node : Every other node is measured based to this one
		\node (q1) [state, black, scale = 1, initial above, initial distance = 5mm, initial text = {Αρχή},
		top color = green!80, bottom color = green!20] {$q_{1}$};

		\node (q2) [state, black, scale = 1, top color = gray!80, bottom color = gray!20, below = of q1]
		{$q_{2}$};

		\node (q3) [state, black, scale = 1, top color = gray!80, bottom color = gray!20, below = of q2]
		{$q_{3}$};

		\node (q4) [state, black, scale = 1, top color = gray!80, bottom color = gray!20, below = of q3]
		{$q_{4}$};

		\node (q5) [state, black, scale = 1, top color = gray!80, bottom color = gray!20, below right = of q4]
		{$q_{5}$};

		% Final Node
		\node (q6) [state, black, scale = 1, accepting, top color = red!80, bottom color = red!20, double
		distance = 3pt, double = red!20, below left = of q4] {$q_{6}$};

		% Draw Arrows/Connections
		\path [-stealth, thick]

		% 0,0 => 1,0 || 0,1
		(q1) edge [] node [left] {$ε,\,ε,\,\$$} (q2)
		% 0,1 => 1,1 || 0,2
		(q2) edge [bend right = 60, looseness = 1.2] node [left] {$ε,\$,ε$} (q6)
		(q2) edge [bend left = 60, looseness = 1.2] node [right] {$a,B,ε$} (q5)
		(q2) edge [] node [left] {$a,\$,A\$$} (q3)
		(q2) edge [] node [right] {$a,B,AB$} (q3)
		(q2) edge [loop right, out = 350, in = 20, looseness = 8] node [right] {$b,\$,B\$$} (q2)
		(q2) edge [loop left, out = 190, in = 160, looseness = 8] node [left] {$b,B,BB$} (q2)
		% 0,2 => 1,2 || 0,3
		(q3) edge [loop right, out = 345, in = 315, looseness = 8] node [right] {$a,A,AA$} (q3)
		(q3) edge [] node [left] {$b,A,ε$} (q4)
		% 0,3 => 1,3 || 0,0
		(q4) edge [loop right] node [right] {$b,A,ε$} (q4)
		(q4) edge [] node [left] {$a,B,ε$} (q5)
		(q4) edge [] node [left] {$ε,\$,ε$} (q6)
		% 1,0 => 0,0 || 1,1
		(q5) edge [loop right] node [right] {$a,B,ε$} (q5)
		(q5) edge [] node [below] {$ε,\$,ε$} (q6);
	\end{tikzpicture}
\end{tcolorbox}

\reducevspace\reducevspace\reducevspace


\begin{tcolorbox}[colback=yellow!15!white, colframe=blue!50!white,
	fonttitle=\bfseries\Large, title = {Υπολογισμός αποδοχής $bbbabaaa$}]
	\reducevspace\reducevspace\reducevspace\reducevspace\reducevspace\reducevspace\reducevspace\reducevspace
	\reducevspace\reducevspace\reducevspace\reducevspace\reducevspace\reducevspace\reducevspace\reducevspace
	\reducevspace\reducevspace\reducevspace\reducevspace\reducevspace\reducevspace\reducevspace\reducevspace
	\begin{adjustwidth}{-0.75em}{}
		\begin{equation*}
			\begin{aligned}
				(q1, bbbabaaa, \varepsilon)
				&\vdash (q2, bbbabaaa, \$) \vdash (q2, bbabaaa, Β\$) \vdash (q2, babaaa, BB\$) \\
				&\vdash (q2, abaaa, BBB\$) \vdash
				\left\{
				\begin{array}{l}
					(q3, baaa, ABBB\$) \\
					(q5, baaa, BB\$) \nvdash \text{Αποτυχία Κλώνου}
				\end{array}
				\right. \\
				&\vdash (q4, aaa, BBB\$) \vdash (q5, aa, BB\$) \vdash (q5, a, B\$) \vdash (q5, \varepsilon, \$)\\
				&\vdash (q6, \varepsilon, \varepsilon)\qquad\quad \text{ Επιτυχής Ανάγνωση}
			\end{aligned}
		\end{equation*}
	\end{adjustwidth}
\end{tcolorbox}
\reducevspace\reducevspace\reducevspace\reducevspace\reducevspace\reducevspace\reducevspace
\reducevspace\reducevspace\reducevspace\reducevspace\reducevspace\reducevspace\reducevspace
\begin{center}
	%\vspace{2em}
	\noindent\rule{\linewidth}{0.5pt}
	%\vspace{2em}
\end{center}
%\clearpage
