\subsection{Απάντηση Υποερωτήματος (α)}
\label{ssec:Solution_2.1}
\doublespacing
Ψάχνουμε από μία συμβολοσειρά όπου αν αναγνωσθεί οδηγεί στην εκάστοτε κατάσταση ισοδυναμίας:

%\hfill \break
\begin{tcolorbox}[colback=yellow!15!white, colframe=blue!50!white,
	fonttitle=\bfseries\Large, title = Αντιπροσοπευτικό στοιχείο για έκαστη κλάση της
	$\approx\mathcal{L}(M)$]

	\centering

	\begin{multicols}{3}
		\begin{itemize}
		\itemsep0em
			\item \textcolor[RGB]{0,100,0}{$\{q_{1}, q_{5}\}$} : $ε$
			\item $\{q_{2}, q_{3}\}$ : $b$
			%%%%
			\item $\{q_{4}, q_{6}\}$ : $baaa$
			\item \textcolor{red}{$\{q_{7}\}$} : $ba$
			\item $\{q_{8}\}$ : $baa$
		\end{itemize}
	\end{multicols}


\end{tcolorbox}
\reducevspace\reducevspace\reducevspace\reducevspace\reducevspace\reducevspace\reducevspace\reducevspace
\begin{center}
	%\vspace{2em}
	\noindent\rule{\linewidth}{0.5pt}
	%\vspace{2em}
\end{center}
%\hfill \break
\clearpage