\section{Άσκηση 5η - Ελαχιστοποίηση Καταστάσεων:}
\label{sec:Exercise_5}
\doublespacing

[25\%]  ΄Εστω το ντετερμινιστικό αυτόματο M που εικονίζεται παρακάτω:

\bm{\textcolor{blue}{(α)}}  Κατασκευάστε αναλυτικά το ισοδύναμο πρότυπο αυτόματο $M'$

\bm{\textcolor{blue}{(β)}} Πόσες κλάσεις ισοδυναμίας έχει κάθε μία από τις παρακάτω σχέσεις; $\sim M,\, \sim M',\\
\approx{L}(M),\, \approx{L}(M')$.

\bm{\textcolor{blue}{(γ)}} Περιγράψτε τις κλάσεις ισοδυναμίας της σχέσης $\sim M$

\bm{\textcolor{blue}{(δ)}} Δώστε ένα αντιπροσωπευτικό στοιχείο της κάθε κλάσης της σχέσης $\approx{L}(M)$.

\hfill \break

\begin{figure}[!htb] \centering
	\begin{tikzpicture} [blue, node distance = 3cm, on grid, font=\sffamily\large\bfseries]
		% Help grid
		%	\draw [help lines] (-1,4) grid (8,-10);
		% Start Node : Every other node is measured based to this one
		\node (q1) [state, black, scale = 1, initial left, initial distance = 4mm, initial text = {Αρχή},
		top color = green!80, bottom color = green!20] {$q_{1}$};

		% Middle Nodes
		\node (q2) [state, black, scale = 1, top color = gray!80, bottom color = gray!20, below = of q1]
		{$q_{2}$};

		\node (q5) [state, black, scale = 1, top color = gray!80, bottom color = gray!20, right = of q1]{$q_{5}$};

		\node (q3) [state, black, scale = 1, top color = gray!80, bottom color = gray!20, right = of q5]{$q_{3}$};

		\node (q4) [state, black, scale = 1, top color = gray!80, bottom color = gray!20, right = of q3]{$q_{4}$};

		\node (q6) [state, black, scale = 1, top color = gray!80, bottom color = gray!20, right = of q2]{$q_{6}$};

		\node (q8) [state, black, scale = 1, top color = gray!80, bottom color = gray!20, below = of q4]{$q_{8}$};

		% Final Node
		\node (q7) [state, black, scale = 1, accepting, top color = red!80, bottom color = red!20,
		double distance = 3pt, double = red!20, right = of q6] {$q_{7}$};

		% Draw Arrows/Connections
		\path [-stealth, thick]
		(q1) edge [loop above] node [left] {$a$}(q1)
		(q1) edge [] node [left] {$b$} (q2)
		(q2) edge [bend right = 40] node [above] {$a$} (q7)
		(q2) edge [] node [left] {$b$} (q5)
		(q3) edge [bend left = 20] node [right] {$a$} (q7)
		(q3) edge [bend left = 20] node [below] {$b$} (q5)
		(q4) edge [bend right = 40] node [below] {$a$} (q5)
		(q4) edge [bend left = 20] node [right] {$b$} (q8)
		(q5) edge [] node [above] {$a$} (q1)
		(q5) edge [bend left = 20] node [above] {$b$} (q3)
		(q6) edge [] node [right] {$a$} (q5)
		(q6) edge [bend right = 40] node [above] {$b$} (q8)
		(q7) edge [] node [above] {$a$} (q6)
		(q7) edge [bend left] node [left] {$b$} (q3)
		(q8) edge [bend left = 20] node [left] {$a$} (q4)
		(q8) edge [out = 55, in = 25, min distance = 6.5cm] node [above] {$b$} (q1);

	\end{tikzpicture}
	%\end{flushleft}
	\captionsetup{labelformat=empty}
	\caption{Αρχικό DFA}
	\label{fig:sub1}
\end{figure}
\clearpage