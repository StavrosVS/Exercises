\subsection{Απάντηση Υποερωτήματος (β)}
\label{ssec:Solution_1.2}
\doublespacing

\par Η μηχανή μας δέχεται ουσιαστικά δέχεται ακριβώς δύο τιμές του δυαδικού συστήματος αρίθμησης, ίσου μήκους
μεταξύ τους και συμπεριλαμβανομένης της κενής λέξης (και άρα καμία τιμής).\\
Οι εγγραφές της ταινίας αρχίζουν και τελειώνουν υποχρεωτικά με κενό καθώς και οι δύο λέξεις διαχωρίζονται με κενό
ανάμεσα τους.\\
Άρα από τα δύο πρώτα ήδη καταλαβαίνουμε ότι αν οι λέξεις μας είναι κενές τότε έχουμε απλά τρία συνεχόμενα κενά στην
ταινία. Αυτό με τη σειρά του σημαίνει ότι έχουμε ουσιαστικά άπειρα κενά δεδομένου ότι η ταινία του θεωρητικού
μοντέλου είναι απείρου μήκους και κάθε θέση δεξιότερα του δεξιότερου (πλην κενού) συμβόλου, αποτελείτε από κενές
θέσεις.\\
Σε αυτή τη περίπτωση ουσιαστικά η έξοδος της μηχανής ισούται ακριβώς με την είσοδο (τρία κενά δίδονται, τρία κενά
λαμβάνονται).\\
Η άλλη περίπτωση είναι οι λέξεις αυτές να έχουν μη μηδενικό μήκος.\\
Στην περίπτωση αυτή η μηχανή θα αντικαταστήσει την αριστερή λέξη με το αποτέλεσμα της λογικής πράξης NAND μεταξύ
των δύο δυαδικών τιμών και την δεξιότερη με το αποτέλεσμα της λογικής πράξης AND μεταξύ τους.

\par
Εφόσον μπορούμε να λάβουμε κενές τιμές τότε βέλτιστο θα ήταν να τις ελέγξουμε εξ αρχής και άρα το πρώτο βήμα που
επιλέγουμε θα είναι ο έλεγχος για κενή πρώτη συμβολοσειρά.\\
Εφόσον όπως είπαμε ήδη οι δύο τιμές θα πρέπει να είναι ίσου μήκους, εάν η πρώτη μας συμβολοσειρά δεν περιέχει
κανένα σύμβολο τότε δεν θα περιέχει ούτε η δεύτερη.\\
Αυτό σημαίνει ότι απλά με μία κίνηση κεφαλής δεξιά και ανάγνωση μπορούμε ήδη να γνωρίζουμε αν η ταινία αποτελείτε
μόνο από κενές θέσεις ή όχι και να δρομολογήσουμε ανάλογα.

\par
Έπειτα παρατηρούμε ότι θα πρέπει να εκτελέσουμε τις λογικές πράξεις AND και NAND και να προβούμε σε αντικατάσταση
των αρχικών τιμών με τα αντίστοιχα αποτελέσματα αυτών των πράξεων.\\
Παρατηρούμε ότι για να εκτελέσουμε τη NAND ουσιαστικά εκτελούμε μία AND και κατόπιν παίρνουμε την άρνηση της, όντως
μία πιο περίπλοκη διαδικασία η μία της άλλης.\\
Άρα βάση αυτού αποφασίζομαι ότι θα ήταν βέλτιστο να υπολογίσουμε μόνο την AND και κατόπιν για την τιμή αποτέλεσμα
της NAND απλά να πάρουμε την άρνηση του προηγούμενου αποτελέσματος, κάνοντας έτσι ουσιαστικά δύο λογικές πράξεις
αντί για τρεις, συνολικά και για τις δύο τιμές μαζί.\\
Ουσιαστικά οι δύο διαδικασίες έχουν κοινό (αρχικό) μέρος και αντί να κατασκευάσουμε και να εκτελέσουμε δύο φορές
τον ίδιο αλγόριθμο, το κάνουμε μία φορά, με έναν υπολογισμό και για τα δύο και απλά προσθέτουμε τον επιπρόσθετο
υπολογισμό στο NAND τμήμα.\\
Τέλος με το πέρας των υπολογισμών και των μετατροπών, η κεφαλή θα πρέπει να μεταφέρεται ξανά πίσω στο πρώτο κενό
(όπου και βρίσκεται κατά την εκκίνηση).

\par
Τώρα παρατηρούμε ότι είναι πιθανότατο να έχουμε πρόβλημα με την απομνημόνευση του πια σύμβολα έχουν
αναγνωσθεί/υπολογισθεί, πια έχουν αντικατασταθεί και πια ακόμη περιμένουν τη σειρά τους. Με λίγα λόγια θα πρέπει να
βρούμε τρόπο ο αλγόριθμος να θυμάται τί έχει περάσει ή πειράξει.\\
Έχουμε σκεφθεί δύο μεθόδους, αλλά όπως πάντα είναι σχεδόν βέβαιο ότι υπάρχουν παραπάνω και επίσης δεν σημαίνει ότι
αυτές που δίδουμε είναι απαραίτητα οι πλέον αποδοτικές. Παρόλα αυτά πιστεύουμε ότι είναι επαρκώς αποδοτικές και
κατανοητές και δεν θα αφιερώσουμε παραπάνω χρόνο στην εύρεση κάτι καλύτερου που δεν γνωρίζουμε κατά πόσο υπάρχει
και με αμφίβολο το αν αξίζει τον επιπλέον κόπο.\\
Η πρώτη μέθοδος που δεν θα ακολουθηθεί τελικά, είναι να γράψουμε τα αποτελέσματα μετά το τρίτο εν σειρά κενό της
ταινίας (δηλαδή πέρα της όλης αρχικής εισόδου που απαιτεί η μηχανή) και κατόπιν να τα μεταφέρουμε (ταυτόχρονα
αντικαθιστώντας το σύμβολο στην θέση τους με κενό χαρακτήρα) επικαλύπτοντας τις αρχικές τιμές (σκεφτείτε το σαν
cut-paste). Με αυτόν τον τρόπο απαλλασσόμαστε μέρους της ανάγκης εισαγωγής πολλών επιπλέον προσωρινών συμβόλων που
θα εκτελούσαν χρέη δεικτών ώστε να γνωρίζουμε μέχρι που και τί έχουμε κάνει.\\
Η μέθοδος αυτή δεν θα προτιμηθεί διότι εάν μία τέτοια διαδικασία υλοποιηθεί, θα απαιτήσει πολλές παραπάνω και
μεγαλύτερες μετακινήσεις της κεφαλή, προκαλώντας έτσι μεγαλύτερες καθυστερήσεις στην παραλαβή της εξόδου και
αυξημένες φθορές στον όλο μηχανισμό (καθώς και επιπλέων καταναλώσεις ενέργειας κοκ). Το ερώτημα δεν μας ζητά να
λάβουμε κάτι τέτοιο υπόψιν και γνωρίζουμε ότι όσο αφορά τις ασκήσεις αυτές μιλάμε ξεκάθαρα για θεωρητικά μοντέλα
και μόνο, παρόλα αυτά θεωρούμε ότι είναι ορθό να σκεφτόμαστε με αυτόν τον τρόπο γενικότερα. Με λίγα λόγια στην δική
μας επίλυση, έμφαση θα δοθεί στην οικονομία κινήσεων της κεφαλής, επιπροσθέτως της ορθής επίλυσης.\\
Η δεύτερη λοιπόν μέθοδος και αυτή που τελικά θα επιλεχθεί, απαιτεί την χρήση παραπάνω χαρακτήρων των ελάχιστων που
απαιτούνται. Αυτοί οι παραπάνω χαρακτήρες θα είναι τουλάχιστον δύο καινούργια σύμβολα, το $T$ και το $F$, το πρώτο
από το True που θα αντικαθιστά τα 1 και το άλλο από το False που θα αντικαθιστά τα 0. Με αυτό το τρόπο πετυχαίνουμε
δύο πράγματα ταυτόχρονα: με ένα και μόνο σύμβολο γνωρίζει η μηχανή και τί έχει αντικατασταθεί ήδη με το αποτέλεσμα
της NAND ή AND και ταυτόχρονα τί τιμή έχει. Ενδεχομένως να χρειαστεί και τρίτο σύμβολο κατά την ανάγνωση ώστε να
θυμάται το σύστημα μέχρι που έχει φτάσει αυτή (η ανάγνωση) όπου σε αυτή τη περίπτωση θα χρησιμοποιήσουμε το σύμβολο
$\$$.\\
Τέλος θα πούμε ότι όπως και στην προηγούμενη άσκηση, έτσι και σε αυτή θα κατασκευάσουμε ανεξάρτητες μηχανές Turing
όπου η κάθε μία θα αναλαμβάνει μέρος του έργου (διαίρει και βασίλευε) και κατόπιν θα τις συνενώνουμε σε μία
παίρνοντας το ζητούμενο.

\par Η μηχανή $M_{S}$ θέλουμε απλά να ελέγχει αν η ταινία είναι κενή (ελέγχοντας αν υπάρχει έστω ένας χαρακτήρας
στη λέξη $x$) και αν ναι να μετακινεί την κεφαλή στη πρώτη θέση μετά την αρχή της ταινίας, η οποία θέση θα πρέπει
απαραίτητα να είναι κενή. Αυτό το πετυχαίνει κάνοντας μία μετακίνηση δεξιά ώστε να βρεθεί η κεφαλή εκεί που
αναμένετε (υποχρεωτικά) το πρώτο σύμβολο της λέξης και κατόπιν ελέγχει εάν αυτό όντως ανήκει στα σύμβολα της λέξης
ή όχι (άρα κενό) και δρομολογεί ανάλογα στην αντίστοιχη επόμενη μηχανή. Τον λόγο για τον οποίο δεν ελέγχουμε και το
$y$ τον έχουμε ήδη αναφέρει και είναι ότι απλά, $x$ και $y$ έχουν ίδιο μήκος και άρα αν το ένα είναι κενό, θα είναι
και το άλλο.
\[\text{Έλεγχος κενής λέξης}\,:\, M_S\,:\quad
L\,\overset{\sqcup}{\longleftarrow}\overset{\lor}{{\text{\large\bfseries
			R}}}\overset{0}{\longrightarrow}\, M_{0LR}\]
\reducevspace\reducevspace\reducevspace\reducevspace\reducevspace\reducevspace\reducevspace\reducevspace\reducevspace
\reducevspace\reducevspace\reducevspace\reducevspace\reducevspace\reducevspace\reducevspace\reducevspace\reducevspace
\reducevspace\reducevspace\reducevspace\reducevspace\reducevspace\reducevspace\reducevspace\reducevspace\reducevspace
\reducevspace\reducevspace\reducevspace\reducevspace\reducevspace\reducevspace\reducevspace\reducevspace\reducevspace
\[\qquad\qquad\qquad\qquad\qquad\qquad\qquad\qquad\quad\;\;\overset{1}{\longrightarrow}\, M_{1LR}\]
\reducevspace\reducevspace\reducevspace\reducevspace\reducevspace\reducevspace\reducevspace\reducevspace\reducevspace
Δοκιμή:
\reducevspace\reducevspace\reducevspace\reducevspace\reducevspace\reducevspace\reducevspace\reducevspace\reducevspace
\begin{itemize}
	\itemsep0em
	\item $|x| = |y| = |z| = |w| = 0 \;\Rightarrow\;\triangleright\, \underline{\sqcup}\, \sqcup \sqcup
	\xrightarrow{R} \triangleright\, \sqcup\, \underline{\sqcup}\, \sqcup \xrightarrow{\sqcup}
	\triangleright\, \underline{\sqcup}\,  \sqcup\, \sqcup  \quad$
	\textcolor{green}{\ding{51}}

	\item{ $|x| = |y| = |z| = |w| = n, n \geq 1 \;\Rightarrow\;\triangleright\, \underline{\sqcup}\, x_1\,...\,y_n
	\sqcup\, y_1\,...\,y_n\, \sqcup \xrightarrow{R}\\ \triangleright\, \sqcup\, \underline{x_1}\,...\,x_n\,
	\sqcup\, y_1\,...\,y_n\, \sqcup \xrightarrow{0} M_{0LR} \quad$ \textcolor{green}{\ding{51}}\\
	\makebox[4.3cm]{\hfill}$\xrightarrow{1} M_{1LR} \quad$ \textcolor{green}{\ding{51}}}
\end{itemize}
\clearpage

\par Η $M_{0LR}$ διαβάζει, υπολογίζει και αντικαθιστά από αριστερά προς τα δεξιά για $x_n\,=\,0$.
\reducevspace\reducevspace\reducevspace\reducevspace\reducevspace\reducevspace\reducevspace\reducevspace\reducevspace
\reducevspace\reducevspace\reducevspace\reducevspace\reducevspace\reducevspace\reducevspace\reducevspace\reducevspace
%\[\text{Υπολογισμός/Αντικατάσταση Αριστερά προς Δεξιά}\,:\, M_{0LR}\,:\]
\[\overset{\lor}{{\text{\large\bfseries T}}}\,R_\sqcup\, \longrightarrow\,
\overset{F,\,T}{\overset{\botcircleleft}{R}}
\overset{0,\,1}{\longrightarrow}\,
F\,R\,\xrightarrow{0} M_{0RL}\]
\reducevspace\reducevspace\reducevspace\reducevspace\reducevspace\reducevspace\reducevspace
\reducevspace\reducevspace\reducevspace\reducevspace\reducevspace\reducevspace\reducevspace
\reducevspace\reducevspace\reducevspace\reducevspace\reducevspace\reducevspace\reducevspace
\reducevspace\reducevspace\reducevspace\reducevspace\reducevspace\reducevspace\reducevspace
\[\qquad\qquad\qquad\qquad\quad\;\;\;\,\xrightarrow{1}\, M_{1RL}\]

\reducevspace\reducevspace\reducevspace\reducevspace\reducevspace\reducevspace\reducevspace\reducevspace\reducevspace
\reducevspace\reducevspace\reducevspace\reducevspace\reducevspace\reducevspace\reducevspace\reducevspace\reducevspace
\reducevspace\reducevspace\reducevspace\reducevspace\reducevspace\reducevspace\reducevspace\reducevspace\reducevspace
\reducevspace\reducevspace\reducevspace\reducevspace\reducevspace\reducevspace\reducevspace\reducevspace\reducevspace
\[\qquad\qquad\qquad\qquad\qquad\quad\;\;\,\xrightarrow{\sqcup}\, R_\sqcup \rightarrow M_T\]
\reducevspace\reducevspace\reducevspace\reducevspace\reducevspace\reducevspace\reducevspace\reducevspace\reducevspace
Δοκιμή:
\reducevspace\reducevspace\reducevspace\reducevspace\reducevspace\reducevspace\reducevspace\reducevspace\reducevspace
\begin{itemize}
	\itemsep0em
	\item $x\,=\,0,\; y\,=c,\; c\in\{0,\,1\} \;\xRightarrow{M_S}\;
	\triangleright\, \sqcup\, \underline{0}\, \sqcup\, c\, \sqcup\,\xrightarrow{TR_\sqcup}\;
	\triangleright\, \sqcup\, T\, \underline{\sqcup}\, c\, \sqcup\,
	\xrightarrow{\overset{F,\,T}{\overset{\botcircleleft}{R}}\xrightarrow{0,\,1}F}\;
	\triangleright\, \sqcup\, T\, \sqcup\, \underline{F}\, \sqcup\, \xrightarrow{R}\;
	\triangleright\, \sqcup\, T\, \sqcup\, F\, \underline{\sqcup}\, \xrightarrow{\xrightarrow{\sqcup}R_\sqcup}\;
	\triangleright\, \sqcup\, T\, \sqcup\, F\, \sqcup\, \underline{\sqcup}\, \xrightarrow{} M_T
	\quad$ \textcolor{green}{\ding{51}}

	\item $x\,=\,0c,\; y\,=cd,\; c\in\{0,\,1\},\,d\in\{0,\,1\} \;\xRightarrow{M_S}\;
	\triangleright\, \sqcup\, \underline{0}\, c\, \sqcup\, c\, d\, \sqcup\,\xrightarrow{TR_\sqcup}\;$\\$
	\triangleright\, \sqcup\, T\, c\, \underline{\sqcup}\, c\, d\, \sqcup\,
	\xrightarrow{\overset{F,\,T}{\overset{\botcircleleft}{R}}\xrightarrow{0,\,1}F}\;
	\triangleright\, \sqcup\, T\, c\, \sqcup\, \underline{F}\, d\, \sqcup\, \xrightarrow{R}\;
	\triangleright\, \sqcup\, T\, c\, \sqcup\, F\, \underline{d}\, \sqcup\, \xrightarrow{0}\; M_{0RL}
	\quad$\textcolor{green}{\ding{51}}\\
	\makebox[10.5cm]{\hfill} $\xrightarrow{1}\; M_{1RL}\quad$ \textcolor{green}{\ding{51}}
	\end{itemize}


\par Η $M_{0RL}$ διαβάζει, υπολογίζει και αντικαθιστά από δεξιά προς τα αριστερά για $y_n\,=\,0$.
\reducevspace\reducevspace\reducevspace\reducevspace\reducevspace\reducevspace\reducevspace\reducevspace\reducevspace
\reducevspace\reducevspace\reducevspace\reducevspace\reducevspace\reducevspace\reducevspace\reducevspace\reducevspace
%\[\text{Υπολογισμός/Αντικατάσταση Δεξιά προς Αριστερά}\,:\, M_{0LR}\,:\]
\[\overset{\lor}{{\text{\large\bfseries F}}}\,L_\sqcup\, \longrightarrow\,
\overset{0,\,1}{\overset{\botcircleleft}{L}}
\overset{F,\,T}{\longrightarrow}\, R\,T\,R\,\xrightarrow{0} M_{0RL}\]
\reducevspace\reducevspace\reducevspace\reducevspace\reducevspace\reducevspace\reducevspace
\reducevspace\reducevspace\reducevspace\reducevspace\reducevspace\reducevspace\reducevspace
\reducevspace\reducevspace\reducevspace\reducevspace\reducevspace\reducevspace\reducevspace
\reducevspace\reducevspace\reducevspace\reducevspace\reducevspace\reducevspace\reducevspace
\[\qquad\qquad\qquad\qquad\qquad\;\;\,\xrightarrow{1}\, M_{1RL}\]

\reducevspace\reducevspace\reducevspace\reducevspace\reducevspace\reducevspace\reducevspace\reducevspace\reducevspace
\reducevspace\reducevspace\reducevspace\reducevspace\reducevspace\reducevspace\reducevspace\reducevspace\reducevspace
\reducevspace\reducevspace\reducevspace\reducevspace\reducevspace\reducevspace\reducevspace\reducevspace\reducevspace
\reducevspace\reducevspace\reducevspace\reducevspace\reducevspace\reducevspace\reducevspace\reducevspace\reducevspace
\[\qquad\qquad\qquad\qquad\qquad\qquad\;\;\xrightarrow{\sqcup}\, R_\sqcup \rightarrow M_T\]
\reducevspace\reducevspace\reducevspace\reducevspace\reducevspace\reducevspace\reducevspace\reducevspace\reducevspace
Δοκιμή:
\reducevspace\reducevspace\reducevspace\reducevspace\reducevspace\reducevspace\reducevspace\reducevspace\reducevspace
\begin{itemize}
	\itemsep0em
	\item $x\,=\,0c,\; y\,=00,\; c\in\{0,\,1\} \;\xRightarrow{M_S}\;
	\triangleright\, \sqcup\, T\, c\, \sqcup\, F\, \underline{0}\, \sqcup\; \xrightarrow{FL_\sqcup}\;
	\triangleright\, \sqcup\, T\, c\, \underline{\sqcup}\, F\, F\, \sqcup\;
	\xrightarrow{\overset{0,\,1}{\overset{\botcircleleft}{L}}\xrightarrow{F,\,T}T}\;
	\triangleright\, \sqcup\, \underline{T}\, c\, \sqcup\, F\, F\, \sqcup\; \xrightarrow{RT}\;
	\triangleright\, \sqcup\, T\, \underline{T}\, \sqcup\, F\, F\, \sqcup\; \xrightarrow{R}\;
	\triangleright\, \sqcup\, T\, T\, \underline{\sqcup}\, F\, F\, \sqcup\;
	\xrightarrow{\xrightarrow{\sqcup}R_\sqcup}\;
	\triangleright\, \sqcup\, T\, T\, \sqcup\, F\, F\, \underline{\sqcup}\; \xrightarrow{}\; M_T
	\quad$ \textcolor{green}{\ding{51}}
	\clearpage
	\item $x\,=\,0cd,\; y\,=00c,\; d\in \{0,\,1\},\; c\in\{0,\,1\},\,d\in\{0,\,1\} \;\xRightarrow{M_S}\\
	\triangleright\, \sqcup\, T\, c\, d\, \sqcup\, F\, \underline{0}\, c\, \sqcup\; \xrightarrow{FL_\sqcup}\;
	\triangleright\, \sqcup\, T\, c\, d\, \underline{\sqcup}\, F\, F\, c\, \sqcup\;
	\xrightarrow{\overset{0,\,1}{\overset{\botcircleleft}{L}}\xrightarrow{F,\,T}T}\;
	\triangleright\, \sqcup\, \underline{T}\, c\, d\, \sqcup\, F\, F\, c\, \sqcup\; \xrightarrow{RT}\;
	\triangleright\, \sqcup\, T\, \underline{T}\, d\, \sqcup\, F\, F\, c\, \sqcup\; \xrightarrow{R}\;
	\triangleright\, \sqcup\, T\, T\, \underline{d}\, \sqcup\, F\, F\, c\, \sqcup\; \xrightarrow{0}\; M_{0LR}
	\quad$ \textcolor{green}{\ding{51}}\\
	\makebox[8.3cm]{\hfill}$\xrightarrow{1}\; M_{1LR} \quad$ \textcolor{green}{\ding{51}}
\end{itemize}



%\clearpage
%\begin{multicols}{2}
\begin{itemize}
	\itemsep0em

	\item Ελάχιστη περίπτωση:\\\reducevspace
	$S \rightarrow aa$
\reducevspace\reducevspace\reducevspace\reducevspace\reducevspace\reducevspace\reducevspace
	\item Ελάχιστα $b$:\\\reducevspace
	$S \rightarrow aaR \,|\, aRa \,|\, Raa$\\\reducevspace
	$R \rightarrow baaa \,|\, abaa \,|\, aaba \,|\, aaab \,|\, \varepsilon$\\
\reducevspace\reducevspace\reducevspace\reducevspace\reducevspace\reducevspace\reducevspace
	\item $|w| = \aleph_0$:\\\reducevspace
	$S \rightarrow aaR \,|\, aRa \,|\, Raa$\\\reducevspace
	$R \rightarrow RbRaaa \,|\, aRbRaa \,|\, aaRbR \,|\, aaaRbR \,|\, \varepsilon$\\
\reducevspace\reducevspace\reducevspace\reducevspace\reducevspace\reducevspace\reducevspace
	\item Σύμπτυξη:\\\reducevspace
	$S \rightarrow RaRaR$\\\reducevspace
	$M \rightarrow bRaaa \,|\, abRaa \,|\, aabRa \,|\, aaaRbR \,|\, \varepsilon$\\
\end{itemize}
%\end{multicols}

Για να καταλάβουμε για πιο λόγο έφυγε η αναδρομή πριν από το $b$ σε όλες τις παραλλαγές του $R$ εκτός την $aaaRbR$
και άρα να υπάρχει κάποια αιτιολόγηση θα πρέπει να δώσουμε κάποια βασικά παραδείγματα:\\

\clearpage
\begin{itemize}
	\itemsep0em

	\item $bb\,aaa\,aaa\,aa \;:\; S\rightarrow RaRaR \rightarrow (baaa)aRaR \rightarrow (bRaaa)aRaR \rightarrow
			(b(bRaaa)aaa)aRaR \rightarrow (b(b\varepsilon aaa)aaa)aRaR \rightarrow (bb\,aaa\,aaa)a\varepsilon a
			\varepsilon \rightarrow bb\,aaa\,aaa\,aa$

	\item $a\,bb\,aaa\,aaa\,a \;:\; S\rightarrow RaRaR \rightarrow \varepsilon aRa\varepsilon \rightarrow
	a(bRaaa)a \rightarrow a(b(bRaaa)aaa)a \rightarrow a(b(b\varepsilon aaa)aaa)a \rightarrow a\,bb\,aaa\,aaa\,a$

	\item $aa\,bbaaa\,aaa \;:\; S\rightarrow RaRaR \rightarrow \varepsilon a\varepsilon aR \rightarrow
	aa(bRaaa) \rightarrow aa(b(bRaaa)aaa) \rightarrow aa(b(b\varepsilon aaa)aaa) \rightarrow aa\,bb\,aaa\,aaa$

	\item $...$

	\item $aa\,aaa\,bb\,aaa \;:\; S\rightarrow RaRaR \rightarrow \varepsilon a\varepsilon aR \rightarrow aa(aaabR)
	\rightarrow aa(aaab(bRaaa)) \rightarrow aa(aaab(b\varepsilon aaa)) \rightarrow aa\,aaa\,bb\,aaa$

	\item $...$

	\item $aaa\,aaa\,bb\,aa \;:\; S\rightarrow RaRaR \rightarrow Ra\varepsilon a\varepsilon \rightarrow
	(aaab)aa \rightarrow (aaaRb)aa \rightarrow (aaa(aaaRb)b)aa \rightarrow (aaa(aaa\varepsilon b)b)aa \rightarrow
	aaa\,aaa\,bb\,aa$

	\item Χρειαζόμαστε $aaaRb$ και $aaabR$. Θυμόμαστε ότι $R$ είναι αναδρομή στον ίδιο κανόνα. Παρατηρούμε ότι και
	στα δύο, είναι ίδια συμβολοσειρά, με αναδρομή πριν ή μετά το $b$ και αντικαθιστούμε με $aaaRbR$.

	\item Για να σιγουρευτούμε, ότι δεν χρειάζονται κανόνες τύπου $"...RbR..."$, έχουμε αναλύσει όλες τις
	παραλλαγές για ορθές συμβολοσειρές με έως τρία $b$, καθώς και κάποιες χαρακτηριστικές μη ορθές. Δεν τις
	συμπεριλαμβάνουμε για προφανείς λόγους. Αυτό δεν σημαίνει ότι δεν γίνεται να μας έχει ξεφύγει κάτι. Υπάρχει και
	καλύτερος τρόπος να δειχθεί γιατί λειτουργεί αυτή η γραμματική αλλά είναι επίσης φλύαρη. Παρακάτω θα δείξουμε
	εναλλακτική γραμματική.

	\item Τέλος αν πούμε ότι η CFG μας δεν περιέχει αριστερή αναδρομή.
\end{itemize}

\par
Γενικά θα πρέπει να έχουμε κατά νου ότι είναι πάρα πολύ χρήσιμο έως και απαραίτητο, εκτός των ελάχιστων περιπτώσεων
βάση συνόλου (πχ $\in \mathbb{N}_0$ ή και ελάχιστων περιπτώσεων για μη μηδενικό σύνολο (ουσιαστικά λαμβάνοντας
υπόψιν αποδοχή κενής συμβολοσειράς εάν επιτρέπεται και ελάχιστης μή κενής), καλό θα είναι να ελέγχουμε και για
διπλές εμφανίσεις των ελάχιστων μή κενών πχ $\{\forall w \neq \varepsilon \,|\, w_{min} = xy,\, \exists w_{double}
= xxyy \cup xyxy \cup xyyx \cup yxxy \cup yxyx \cup yyxx \}$

\par
Υπόψιν ότι εκτός κάποιον πολύ απλών περιπτώσεων, στις CFL δεν μπορούμε πάντα να είμαστε σίγουροι ότι η CFG μας
καλύπτει όλες τις πιθανές ορθές συμβολοσειρές της γλώσσας ή ότι είναι ελάχιστη.

\par
Θα δώσω επίσης άλλη μία λύση εν συντομία μία και έχει ήδη δειχθεί (απλά απλοποιήθηκε):

\begin{itemize}
	\item Η παρακάτω απλή και εύκολη γραμματική παρόλο που καλύπτει την γλώσσα, χρησιμοποιεί αριστερή αναδρομή κάτι
	που ενώ δεν αναφέρεται ως περιορισμός στην άσκηση, θα πρέπει να γνωρίζουμε ότι αναλόγως τα εργαλεία που θα
	χρησιμοποιήσουμε για την υλοποίηση, ίσως να προκύψει θέμα "συμβατότητας" (και είναι ένας λόγος που την
	αποφύγαμε, επιπροσθέτως στο ότι αυτή που προαναφέραμε αποτελεί ελαχιστοποίηση αυτής και άνευ αριστερής
	αναδρομής). Το πως καταλήξαμε σε αυτή φαίνεται στα βήματα της προηγούμενης προ σύμπτυξης:\\
	$G_{2b} = (\{a, b\},\, \{aa, ε\},\, \{S\rightarrow AaAaA,\,
	A\rightarrow Raaa \,|\, aRaa \,|\, aaRa \,|\, aaaR \,|\, \varepsilon,\,\\
	R\rightarrow AbA\},\, S)$
\end{itemize}

\hfill \break

\begin{tcolorbox}[colback=yellow!15!white, colframe=blue!50!white,
	fonttitle=\bfseries\Large, title = Γραμματική και συντακτικό δένδρο]
	Είμαστε έτοιμοι να προχωρήσουμε στην πλήρη περιγραφή της γραμματικής μας:\\
	$G_2 = (\{a, b\},\, \{aa, ε\},\, \{S\rightarrow RaRaR\\
	R\rightarrow bRaaa \,|\, abRaa \,|\, aabRa \,|\, aaaRbR \,|\, \varepsilon\},\, S)$

	Ήρθε η στιγμή να δώσουμε το συντακτικό δένδρο για συμβολοσειρά (τα κενά για ευκολότερη ανάγνωση) $w =
	aa\,b\,aaa\,b\,aaa,\,|w| = 10,\, |w|_a = 8,\,|w|_b = 2$:


	\begin{center}
		\Tree
		[.{S}
			[.{R}
				[.{$aab$} ]
				[.{R}
					[.{$\varepsilon$} ]
				]
				[.{$a$} ]
			]
			[.{$a$} ]
			[.{R}
				[.{$ab$} ]
				[.{R}
					[.{$\varepsilon$} ]
				]
				[.{$aa$} ]
			]
			[.{$a$} ]
			[.{R}
				[.{$\varepsilon$} ]
			]
		]
	\end{center}


	Που δίνει:$\qquad\qquad\quad\; aab\qquad\;\, a\;\;\; a\;\;\; ab\qquad\; aa\;\;\; a$\\
	Εναλλακτικά θα μπορούσαμε να το διαβάσουμε ως $(\varepsilon)\,a\,(ab(\varepsilon) aa)\,a\,(b(\varepsilon)aaa)$.

\end{tcolorbox}


\begin{center}
	%\vspace{2em}
	\noindent\rule{\linewidth}{0.5pt}
	%\vspace{2em}
\end{center}
\clearpage