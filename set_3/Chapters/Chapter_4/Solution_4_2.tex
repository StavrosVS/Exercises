\subsection{Απάντηση Υποερωτήματος (β)}
\label{ssec:Solution_4.2}
\doublespacing

Χρησιμοποιούμαι το παρακάτω ως αναφορά για τους κανόνες παραγωγής καθώς με αρίθμηση για να μπορούμε να διαχωρίσουμε
αυτούς που ξεκινάνε από ίδιο μη τερματικό σύμβολο:

\begin{multicols}{2}
	\begin{enumerate}
		\item $S\rightarrow LX_4$
	    \item $S\rightarrow LX_3$
		\item $S\rightarrow X_1X_2$
		\item $S\rightarrow D_aD$
		\item $S\rightarrow DD_a$
		\item $S\rightarrow \varepsilon$

		\item $A\rightarrow LX_4$
		\item $A\rightarrow LX_3$
		\item $A\rightarrow X_1X_2$
		\item $A\rightarrow D_aD$

		\item $L\rightarrow DD_a$

		\item $D\rightarrow b$

		\item $D_a\rightarrow a$

		\item $X_1\rightarrow DA$

		\item $X_2\rightarrow DD_a$

		\item $X_3\rightarrow D_aD$

		\item $X_4\rightarrow X_1X_2$
	\end{enumerate}
\end{multicols}
\vfill
\clearpage



\begin{tcolorbox}[colback=yellow!15!white, colframe=blue!50!white,
	fonttitle=\bfseries\Large, title = {Πίνακας συντακτικής ανάλυσης για $w = babba$}]

\begin{itemize}
	\itemsep0em
	\item Κατασκευάζουμε πίνακα $|w|$ στηλών και για κάθε στήλη $n = \in [1,\, |w|_{max}]$ ξεκινώντας να μετράμε
	από αριστερά ως 1η, έχουμε $n$ κελιά.

	\item Στο κορυφαίο κελί κάθε στήλης (δηλαδή το $n$), τοποθετούμε το αντίστοιχο σύμβολο της λέξης (δηλαδή το
	$w_n$).
	\item Κάθε ένα από αυτά τα σύμβολα ουσιαστικά αποτελεί κανόνα παραγωγής τερματικού συμβόλου.

	\item Ανά στήλη, σε κελί $k,\, k \in [2,\, n]$ με $k_{max}$ να βρίσκεται στη βάση του πίνακα και
	$k_{min}$ ακριβώς στο κελί κάτω από αυτό στην κορυφή της στήλης του, τοποθετούμε τους κανόνες παραγωγής (τα μη
	τερματικά σύμβολα) τα οποία δύνανται να οδηγήσουν σε αριστερότερο τμήμα της $w$ μήκους υποσυμβολοσειράς $x$ με
	$|x| = k,\, x= \displaystyle \sum_{i=1}^{k} w_i $

	\item Όταν ολοκληρώσουμε περιμένουμε να βρούμε στη κάτω δεξιά γωνία κάποια κανόνα παραγωγής, από το αρχικό (μη
	τερματικό προφανώς) σύμβολο. Εάν αυτό υπάρχει τότε $w \in G$, ειδάλλως $w \notin G$.

\end{itemize}

	\reducevspace\reducevspace\reducevspace\reducevspace\reducevspace\reducevspace
	\reducevspace\reducevspace\reducevspace\reducevspace\reducevspace\reducevspace
	\reducevspace\reducevspace\reducevspace\reducevspace\reducevspace\reducevspace
	\reducevspace\reducevspace\reducevspace\reducevspace\reducevspace\reducevspace
	\reducevspace\reducevspace\reducevspace\reducevspace\reducevspace\reducevspace
	\reducevspace\reducevspace\reducevspace\reducevspace\reducevspace\reducevspace
\begin{center}
\resizebox{\textwidth}{!}{%
	\begin{tabular}{llll|l|}
		\cline{5-5}
		&                       &                        &      & a                                 \\ \cline{4-5}
		&                       & \multicolumn{1}{l|}{}  & b    & L{[}11{]}, $X_2${[}15{]}, S{[}5{]} \\ \cline{3-5}
		& \multicolumn{1}{l|}{} & \multicolumn{1}{l|}{b} & $\emptyset$ & $\emptyset$
		\\ \cline{2-5}
		\multicolumn{1}{l|}{} &
		\multicolumn{1}{l|}{a} &
		\multicolumn{1}{l|}{S{[}4{]}, A{[}10{]}, $X_3${[}16{]}} &
		$\emptyset$ &
		$\emptyset$ \\ \hline
		\multicolumn{1}{|l|}{b} &
		\multicolumn{1}{l|}{L{[}11{]}, $X_2${[}15{]}, S{[}5{]}} &
		\multicolumn{1}{l|}{$X_1${[}14{]}} &
		$\emptyset$ &
		A{[}9{]}, S{[}3{]}, $X_4${[}17{]} \\ \hline
	\end{tabular}%
}
\end{center}
\end{tcolorbox}
\clearpage

\begin{tcolorbox}[colback=yellow!15!white, colframe=blue!50!white,
	fonttitle=\bfseries\Large, title = {Δέντρο συντακτικής ανάλυσης για $w = babba$}]

	\begin{itemize}
		\itemsep0em
		\item Κατασκευάζουμε δέντρο συντακτικής ανάλυσης βάση αντίστροφης χρήσης των αποτελεσμάτων του πίνακα,
		δηλαδή ξεκινώντας από την κάτω δεξιά γωνία και συγκεκριμένα την αντίστοιχη εκδοχή αρχικού σύμβολου εφόσον η
		$w$ όντως ανήκει στη γραμματική μας.

		\item Σε κάθε βήμα ακολουθούμε κλάδο βάση κανόνων παραγωγής αυτής της εκδοχής συμβόλου και ακολουθείτε από
		την εκάστοτε εκδοχή επόμενων με τερματικών συμβόλων που έχουν βρεθεί στο αντίστοιχο βήμα στο πίνακα, έως να
		φτάσουμε διαδοχικά στα κατάλληλα τερματικά σύμβολα που κατασκευάζουν τη λέξη που μας δόθηκε.

	\end{itemize}

	\begin{center}
	\Tree
	[.{S[3]}
		[.{$X_{1}$[14]}
			[.{$D$[12]}
				[.{$b$} ]
			]
			[.{$A$[10]}
				[.{$D_a$[13]}
					[.{$a$} ]
				]
				[.{$D$[12]}
					[.{$b$} ]
				]
			]
		]
		[.{$X_{2}$[15]}
			[.{$D$[12]}
				[.{$b$} ]
			]
			[.{$D_a$[13]}
				[.{$a$} ]
			]
		]
	]
\end{center}


Που δίνει:$\qquad\qquad\quad b\qquad\quad a\qquad\quad b\qquad\;\;\; b\qquad\quad a$
\end{tcolorbox}

\vfill

\begin{center}
	%\vspace{2em}
	\noindent\rule{\linewidth}{0.5pt}
	%\vspace{2em}
\end{center}
\clearpage