\subsection{Απάντηση Υποερωτήματος (α)}
\label{ssec:Solution_4.1}
\doublespacing

\reducevspace\reducevspace\reducevspace\reducevspace\reducevspace\reducevspace\reducevspace
Ο ισχυρισμός είναι λανθασμένος και ως απόδειξη δείχνοντας έστω μία περίπτωση όπου ένωση μετρήσιμης με μή κανονικής
γλώσσας παράγει μή κανονική γλώσσα. Αφού ο ισχυριζόμαστε ότι από $\mathcal{L}_{fin} \cup \mathcal{L}_{irr}$
\textbf{ΠΑΝΤΑ} παράγεται κανονική άρα εάν βρεθεί
οποιαδήποτε εξαίρεση ο ισχυρισμός απορρίπτεται ως λανθασμένος.

%\hfill \break
\begin{tcolorbox}[colback=yellow!15!white, colframe=blue!50!white,
	fonttitle=\bfseries\Large, title = Απόδειξη - μέρος 1/3]
	\begin{itemize}
		\itemsep1em

		\item Ισχύει $\mathcal{L}_{fin} \subset \mathcal{L}_{reg} \overset{\mathcal{L}_{reg} =
	 	\mathcal{L}_{irr}^c}{\rightarrow} \mathcal{L}_{fin} \cap \mathcal{L}_{irr} = \emptysetAlt$
	 	\reducevspace\reducevspace\reducevspace\reducevspace\reducevspace\reducevspace\reducevspace
	 	\reducevspace\reducevspace\reducevspace\reducevspace\reducevspace\reducevspace\reducevspace
	 	\reducevspace\reducevspace\reducevspace\reducevspace\reducevspace\reducevspace\reducevspace
		\begin{flushright}\hypertarget{4.1.1}{\bf{(1)}}\end{flushright}

		\item Ο ισχυρισμός του ερωτήματος ισοδυναμεί με
		$\overset{\hyperlink{4.1.1}{\bf{\textcolor{red}{(1)}}}}{\rightarrow}$ $\mathcal{P} := $ "Η ένωση
		κανονικής γλώσσας
		με μή κανονικής είναι πάντα κανονική γλώσσα" $\rightarrow$\\ $\mathcal{P} :=
		\{\forall\; L_{reg} \cup L_{irr} \,\vert\,
		(L_{reg} \in \mathcal{L}_{reg},\, L_{irr} \in \mathcal{L}_{irr})\}\in \mathcal{L}_{reg}$.
		\reducevspace\reducevspace\reducevspace\reducevspace\reducevspace\reducevspace\reducevspace
		\reducevspace\reducevspace\reducevspace\reducevspace\reducevspace\reducevspace\reducevspace
		\reducevspace\reducevspace\reducevspace\reducevspace\reducevspace\reducevspace\reducevspace
		\begin{flushright}\hypertarget{4.1.2}{\bf{(2)}}\end{flushright}

		\begin{comment}
		\item Οι κανονικές γλώσσες είναι κλειστές ως προς την πράξη της διαφοράς όπως αποδεικνύουμε στην απάντηση
		του υποερωτήματος 4-β και συγκεκριμένα στην 2η απόδειξη.
		\begin{flushright}(5)\end{flushright}
		\end{comment}

		\item $\mathcal{Q} := $"Υπάρχει τουλάχιστον μία πεπερασμένη γλώσσα με μία μή
		κανονική γλώσσα με την ένωση τους να μην είναι κανονική
		γλώσσα"$\overset{\hyperlink{4.1.1}{\bf{\textcolor{red}{(1)}}}}{\rightarrow}$\\
		$\mathcal{Q} :=
		\{\exists\ L_{reg} \cup L_{irr} \,\vert\,
		(L_{reg} \in \mathcal{L}_{reg},\, L_{irr} \in \mathcal{L}_{irr})\}\in \mathcal{L}_{irr}$.
		\reducevspace\reducevspace\reducevspace\reducevspace\reducevspace\reducevspace\reducevspace
		\reducevspace\reducevspace\reducevspace\reducevspace\reducevspace\reducevspace\reducevspace
		\reducevspace\reducevspace\reducevspace\reducevspace\reducevspace\reducevspace\reducevspace
		\begin{flushright}\hypertarget{4.1.3}{\bf{(3)}}\end{flushright}

		\item Γλώσσα όπου εφαρμογή λήμματος άντλησης οδηγεί σε άτοπο είναι μή κανονική.
		\reducevspace\reducevspace\reducevspace\reducevspace\reducevspace\reducevspace\reducevspace
		\reducevspace\reducevspace\reducevspace\reducevspace\reducevspace\reducevspace\reducevspace
		\reducevspace\reducevspace\reducevspace\reducevspace\reducevspace\reducevspace\reducevspace
		\reducevspace\reducevspace\reducevspace\reducevspace\reducevspace\reducevspace\reducevspace
		\reducevspace\reducevspace\reducevspace\reducevspace\reducevspace\reducevspace\reducevspace
		\begin{flushright}\hypertarget{4.1.4}{\bf{(4)}}\end{flushright}


		\item Ως λήμμα άντλησης$^{\hyperlink{4.1.4}{\bf{\textcolor{red}{(4)}}}}$ κανονικής γλώσσας $L$ ορίζεται η
		εξής μέθοδος:\\
		$\{L\subseteq \Sigma^* \mid \exists n \geq 1 \ni (\forall w \in
		L,\, \vert w\vert \geq n \ni $
		\reducevspace\reducevspace\reducevspace\reducevspace\reducevspace\reducevspace\reducevspace
		\reducevspace\reducevspace\reducevspace\reducevspace\reducevspace\reducevspace\reducevspace
		\begin{flushright}$[\exists x,y,z \in \Sigma^* \ni ( w = xyz,\, y \neq \mathcal{ε},\,
			\vert xy\vert \leq n) \rightarrow ( i \geq 0 \ni \exists xy^iz \in L)])\}$\end{flushright}
		\reducevspace\reducevspace\reducevspace\reducevspace\reducevspace\reducevspace\reducevspace
		\reducevspace\reducevspace\reducevspace\reducevspace\reducevspace\reducevspace\reducevspace
		\reducevspace\reducevspace\reducevspace\reducevspace\reducevspace\reducevspace\reducevspace
		\reducevspace\reducevspace\reducevspace\reducevspace\reducevspace\reducevspace\reducevspace
		\begin{flushright}\hypertarget{4.1.5}{\bf{(5)}}\end{flushright}

		\begin{comment}
		\item Ορίζουμε $\mathcal{L}_{regular}$ ως κανονική και $\mathcal{L}_{iregular}$ ως μη κανονική γλώσσα
		και\\
		$\mathcal{L}_{union} = \mathcal{L}_{regular} \cup \mathcal{L}_{irregular}$\\
		Θα υποθέσουμε ότι $\mathcal{P} = "\mathcal{L}_{union}$ είναι πάντα κανονική" όπως δηλαδή μας λέει ο
		ισχυρισμός.
		\end{comment}

		\begin{comment}
		\item Για σύνολα A, Β και C έχουμε\\
		$(Α \cup B) - C = (A \cup B) \cap C^c = (A \cap C^c) \cup (B \cap C^c) = (A-C) \cup (B-C)$ \\
		επιβεβαιώνουμε:\;
		$x \in [(A \cup B) - C] \Longleftrightarrow (x \in A \lor x \in B) \land x \notin C \Longleftrightarrow$\\
		 $(x\in A \lor x \in B) \land x \in C^c = (A) \overset{\text{επιμεριστική}}{\Longleftrightarrow}
		(x \in A \land x \in C^c) \lor (x \in B \land x \in C^c) \Longleftrightarrow$\\
		$(x \in A \land x \notin C)\lor (x \in B \land x \notin C) \Longleftrightarrow
		x \in (A-C) \lor x \in (B-C) \Longleftrightarrow$\\
		$x \in [(A-C) \cup (B-C)]$ άρα $(A \cup B) - C = (A-C) \cup (B-C)$
		\begin{flushright}(6)\end{flushright}


		\item Για $A = C \overset{\text{(6)}}{\rightarrow} (A \cup B) - A = (A-A) \cup (B-A) = \emptyset \cup (B-A)
		= B-A$
		\begin{flushright}(7)\end{flushright}
		\end{comment}

	\end{itemize}
\end{tcolorbox}

%\hfill \break
\begin{tcolorbox}[colback=yellow!15!white, colframe=blue!50!white,
	fonttitle=\bfseries\Large, title = Απόδειξη - μέρος 2/3]
	\begin{itemize}
		\itemsep1em

		\item \quad Θεωρούμε ότι έχουμε DFA $M$ με
		$M = (K, \Sigma, \Delta, s, F)=(\{q0, q1\}, \{a,\,b\},$\\
		$\{$ δ$(q0, a)\!=\! q1,$ δ$(q0, b)\!=\! q1,$ δ$(q1, a)\!=\! q1,$ δ$(q1, b)\!=\!
		q1 \}, q0,\{q0\})$\\
		που αναγνωρίζει τη γλώσσα $L(M) (\text{εξ ορισμού κανονική})= L_{fin} \in \mathcal{L}_{fin}$.\\
		\reducevspace\reducevspace\reducevspace\reducevspace\reducevspace\reducevspace
		\reducevspace\reducevspace\reducevspace\reducevspace\reducevspace\reducevspace
		\reducevspace\reducevspace\reducevspace\reducevspace\reducevspace\reducevspace
		\reducevspace\reducevspace\reducevspace\reducevspace\reducevspace\reducevspace
		\par \quad Η γλώσσα μας είναι $L_{fin} = \{ w \in \{\mathcal{ε}\}\}$, προφανέστατα πεπερασμένη
		αφού $\vert L\vert = 1 \leq \aleph_0$. Συνεπάγεται ότι
		είναι και κανονική$^{\hyperlink{4.1.1}{\bf{\textcolor{red}{(1)}}}}$ οπότε ορίζουμε αντίγραφο της με σύμβολο
		$L_{reg} \in \mathcal{L}_{reg} =
		L_{fin}$.
		\reducevspace\reducevspace\reducevspace\reducevspace\reducevspace\reducevspace\reducevspace
		\reducevspace\reducevspace\reducevspace\reducevspace\reducevspace\reducevspace\reducevspace
		\reducevspace\reducevspace\reducevspace\reducevspace\reducevspace\reducevspace\reducevspace
		\begin{flushright}\hypertarget{4.1.6}{\bf{(6)}}\end{flushright}

		\item Θεωρούμε γλώσσα $L={w \in (a^x, b^x) : x \in \mathbb{N}}$ (η οποία είναι προφανώς μετρήσιμα
		άπειρη)	και ορίζουμε την
		$L_{irr} = L$. Εφαρμόζουμε το θεώρημα άντλησης$^{\hyperlink{4.1.5}{\bf{\textcolor{red}{(5)}}}}$ υποθέτοντας
		ότι η γλώσσα μας
		είναι κανονική ώστε να αποδείξουμε με εις άτοπον απαγωγή ότι είναι μή κανονική:\\
		$w \in L_{irr},\, x=n,\, w = a^n b^n,\, \vert w\vert = 2n \geq n$
		$^{(\text{βλ}\; \vert w\vert\geq n\; \hyperlink{4.1.5}{\bf{\textcolor{red}{(5)}}})}$,
		$w = xyz,\, x=a^j,$\\
		$y=a^k,\, y \neq \emptyset \rightarrow k \geq 1,\, \vert xy\vert = j+k \leq n,\, z = a^{n-j-k}b^n$\\
		$w = xy^iz = a^j a^ik a^{n-j-k} b^n \Longrightarrow a^{n-j-k+j+ik} b^n \Longrightarrow a^{n-k+jk} b^n$
		$\Longrightarrow n-k+jk = n \Longrightarrow jk-k = 0 \Longrightarrow jk=k \Longrightarrow j = 1$\\
		Άρα για οποιοδήποτε $j \neq 1$ οδηγούμαστε σε άτοπο για παράδειγμα:\\
		$i = 2 \rightarrow w = xy^2z = a^{n-k+2k} b^n \Longrightarrow a^{n+k} b^n$ αλλά $k \geq 1$
		αφού $\vert y \vert \neq \emptyset$ τότε $n+k \neq n$ δηλαδή $\vert w\vert_a \neq \vert w\vert_b$ άρα
		αποδείξαμε ότι η $L_{irr} \in \mathcal{L}_{irr}$ $^{\hyperlink{4.1.4}{\bf{\textcolor{red}{(4)}}}}$.
		\reducevspace\reducevspace\reducevspace\reducevspace\reducevspace\reducevspace\reducevspace
		\reducevspace\reducevspace\reducevspace\reducevspace\reducevspace\reducevspace\reducevspace
		\reducevspace\reducevspace\reducevspace\reducevspace\reducevspace\reducevspace\reducevspace
		\begin{flushright}\hypertarget{4.1.7}{\bf{(7)}}\end{flushright}

		\item $L_{union} = L_{reg} \cup L_{irr}
		\overset{\hyperlink{4.1.7}{\bf{\textcolor{red}{(7)}}}\land\hyperlink{4.1.6}{\bf{\textcolor{red}{(6)}}}}{\Longrightarrow}
		 L_{union} = \{\mathcal{ε}\} \,\cup\, \{a^n b^n : n \in
		\mathbb{N}^0\}$\\
		αλλά $\{\mathcal{ε}\} = \{a^0 b^0\} \overset{n \in \mathbb{N}^0}{\rightarrow} \{\mathcal{ε}\} \in \{a^n
		b^n\ : n \in
		\mathbb{N}^0\}$ και άρα\\
		$L_{union} = \{a^n b^n \vert n \in \mathbb{N}^0\} = L_{irr} \neq$ κανονική γλώσσα
		$\rightarrow \mathcal{Q}=T$.
		\reducevspace\reducevspace\reducevspace\reducevspace\reducevspace\reducevspace\reducevspace
		\reducevspace\reducevspace\reducevspace\reducevspace\reducevspace\reducevspace\reducevspace
		\reducevspace\reducevspace\reducevspace\reducevspace\reducevspace\reducevspace\reducevspace
		\begin{flushright}\hypertarget{4.1.8}{\bf{(8)}}\end{flushright}

	\end{itemize}
\end{tcolorbox}

%\hfill \break
\begin{tcolorbox}[colback=yellow!15!white, colframe=blue!50!white,
	fonttitle=\bfseries\Large, title = Απόδειξη - μέρος 3/3]
	\begin{itemize}
		\itemsep1em



		\item $\mathcal{P} = \neg{\mathcal{Q}}$ αλλά ήδη
		αποδείξαμε$^{\hyperlink{4.1.8}{\bf{\textcolor{red}{(8)}}}}$ ότι $\mathcal{Q}=T$ και άρα:\\
		$\mathcal{P} = \neg{\mathcal{Q}} \Longrightarrow \mathcal{P} = \neg{T} \Longrightarrow \mathcal{P} = F$\\
		Άρα η πρόταση του αρχικού ισχυρισμού αποδείχθηκε λανθασμένη.

	\begin{comment}
		\item $\mathcal{L}_{union} = \mathcal{L}_{regular} \cup \mathcal{L}_{irregular} \Longrightarrow$
		$\mathcal{L}_{union} = (\mathcal{L}_regular \cup \mathcal{L}_irregular) - ()$

		\item Γνωρίζουμε από τις ιδιότητες των κανονικών γλωσσών ότι είναι κλειστές ως προς την πράξη της ένωσης
		από το οποίο και συνεπάγετε επίσης ότι οποιαδήποτε κανονική γλώσσα γίνεται να αποδομηθεί σε δύο κανονικές
		γλώσσες η ένωση των οποίων την αποτελεί και άρα η ένωση $\mathcal{L}_{irregular}$
	\end{comment}


	\end{itemize}
\end{tcolorbox}
\begin{center}
	%\vspace{2em}
	\noindent\rule{\linewidth}{0.5pt}
	%\vspace{2em}
\end{center}
%\clearpage