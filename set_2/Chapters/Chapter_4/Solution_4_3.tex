\subsection{Απάντηση Υποερωτήματος (γ)}
\label{ssec:Solution_4.2}
\doublespacing

Ο ισχυρισμός ότι η διαφορά δύο κανονικών γλωσσών μπορεί να παράξει μη μετρήσιμο πλήθος συμβολοσειρών είναι
λανθασμένος. Θα προχωρήσω σε απόδειξη με δύο μεθόδους:

 %\hfill \break
\begin{tcolorbox}[colback=yellow!15!white, colframe=blue!50!white,
	fonttitle=\bfseries\Large, title = Απόδειξη - μέρος 1/2]
\begin{itemize}
	\itemsep1em

	\item Ως συμβολοσειρά ορίζεται ακολουθία συμβόλων πεπερασμένου μήκους.
	\reducevspace\reducevspace\reducevspace\reducevspace\reducevspace\reducevspace\reducevspace
	\reducevspace\reducevspace\reducevspace\reducevspace\reducevspace\reducevspace\reducevspace
	\begin{flushright}\hypertarget{4.3.1}{\bf{(1)}}\end{flushright}

	\item Ως αλφάβητο $\Sigma$ ορίζεται ως πεπερασμένο σύνολο συμβόλων.
	\reducevspace\reducevspace\reducevspace\reducevspace\reducevspace\reducevspace\reducevspace
	\reducevspace\reducevspace\reducevspace\reducevspace\reducevspace\reducevspace\reducevspace
	\begin{flushright}\hypertarget{4.3.2}{\bf{(2)}}\end{flushright}

	\item $\Sigma^n$ με $n \in \mathbb{N}^0$ να εννοείται το σύνολο των
	συμβολοσειρών$^{\hyperlink{4.3.1}{\bf{\textcolor{red}{(1)}}}}$ μήκους $n$ πχ για γλώσσα
	$\Sigma = {a, b},\, \Sigma^0 = {\emptyset},\, \Sigma^1 = {a, b},\, \Sigma^2 = {aa, ab, ba, bb}$ κοκ.
	Eίναι μετρήσιμο$^{\hyperlink{4.3.2}{\bf{\textcolor{red}{(2)}}}}$ με πλήθος $\vert\Sigma^n\vert = 2^n$.
	\reducevspace\reducevspace\reducevspace\reducevspace\reducevspace\reducevspace\reducevspace
	\reducevspace\reducevspace\reducevspace\reducevspace\reducevspace\reducevspace\reducevspace
	\begin{flushright}\hypertarget{4.3.3}{\bf{(3)}}\end{flushright}

	\item Ορίζουμε σύνολο συμβολοσειρών $\Sigma^*$ αλφάβητου $\Sigma$ για το οποίο ισχύει:\\
	$\{ \Sigma^* \;\vert\; (n \in \mathbb{N}^0)\,[\Sigma^* = \Sigma^0 \cup \Sigma^1 \cup ... \cup \Sigma^n] \}$ \\
	Υπερσύνολο$^{\hyperlink{4.3.3}{\bf{\textcolor{red}{(3)}}}}$ που αποτελείτε από ένωση μετρήσιμων
	συνόλων δεν μπορεί παρά να είναι και το
	ίδιο μετρήσιμο. Επίσης αφού $n \in \mathbb{N}^0$ τότε $\Sigma^*$ μετρήσιμο και με βάση την αρχή του
	περιστερώνα αφού ισχύει\\
	 $\{\{n \in \mathbb{N}^0 \;\vert\; k \in \mathbb{N}\}:[\exists n_k : k = n+1]\}$.
	 \reducevspace\reducevspace\reducevspace\reducevspace\reducevspace\reducevspace\reducevspace
	 \reducevspace\reducevspace\reducevspace\reducevspace\reducevspace\reducevspace\reducevspace
	 \begin{flushright}\hypertarget{4.3.4}{\bf{(4)}}\end{flushright}

\end{itemize}
\end{tcolorbox}


\begin{tcolorbox}[colback=yellow!15!white, colframe=blue!50!white,
	fonttitle=\bfseries\Large, title = Απόδειξη - μέρος 2/2]
	\begin{itemize}
		\itemsep1em


	\item Για μία κανονική γλώσσα $\mathcal{L}$ ισχύει $\mathcal{L} \subseteq \Sigma^* = \bigcup\limits_{k =
		0}^{\infty}\Sigma^k$ (γνήσιο υποσύνολο στην -τυπική κατά τα άλλα- περίπτωση που κάποιες συμβολοσειρές δεν
	αναγνωρίζονται λόγο των γραμματικών κανόνων της γλώσσας). Άρα οι συμβολοσειρές μιας κανονικής γλώσσας είναι
	πάντα μετρήσιμες ως υποσύνολο του μετρήσιμου $\Sigma^*$.
	\reducevspace\reducevspace\reducevspace\reducevspace\reducevspace\reducevspace\reducevspace
	\reducevspace\reducevspace\reducevspace\reducevspace\reducevspace\reducevspace\reducevspace
	\begin{flushright}\hypertarget{4.3.5}{\bf{(5)}}\end{flushright}

	\item Ισχύει για σύνολα Α και Β ότι $Α - Β = \{x \vert [x \in A] \land (x \notin B)\}$ και άρα $(A - B)
	\subseteq A$ ενώ $(A - B) \not\subseteq B$
	\reducevspace\reducevspace\reducevspace\reducevspace\reducevspace\reducevspace\reducevspace
	\reducevspace\reducevspace\reducevspace\reducevspace\reducevspace\reducevspace\reducevspace
	\begin{flushright}\hypertarget{4.3.6}{\bf{(6)}}\end{flushright}

	\item Για κανονικές γλώσσες $L_1 \in \mathcal{L}_{reg}$ και $L_2 \in \mathcal{L}_{reg}$
	έχουμε ότι\\
	$L_1 - L_2 =  L_3$ με $L_3 \subseteq L_1^{\hyperlink{4.3.6}{\bf{\textcolor{red}{(6)}}}}$ \\
	και αφού $L_1 \in \mathcal{L}_{reg}$ τότε και η $L_3$ μόνο μετρήσιμη θα μπορούσε να
	είναι$^{\hyperlink{4.3.5}{\bf{\textcolor{red}{(5)}}}}$.
	\reducevspace\reducevspace\reducevspace\reducevspace\reducevspace\reducevspace\reducevspace
	\reducevspace\reducevspace\reducevspace\reducevspace\reducevspace\reducevspace\reducevspace
	\begin{flushright}\hypertarget{4.3.7}{\bf{(7)}}\end{flushright}

	\item Πρόταση $\mathcal{P} =$ "Η διαφορά δύο κανονικών γλωσσών μπορεί να παράξει μη μετρήσιμο πλήθος
	συμβολοσειρών"\\
	Πρόταση $\mathcal{Q} =$ "Η διαφορά δύο κανονικών γλωσσών πάντα παράγει μετρήσιμο πλήθος συμβολοσειρών"\\
	$\mathcal{P} = \neg\mathcal{Q}$ αλλά έχουμε ήδη αποδείξει ότι το $\mathcal{Q}$ είναι
	αληθές$^{\hyperlink{4.3.7}{\bf{\textcolor{red}{(7)}}}}$ δλδ $\mathcal{Q} = Τ$ οπότε $\mathcal{P} = \neg Τ$ και
	άρα $\mathcal{P} = F$ και άρα ο αρχικός ισχυρισμός είναι λανθασμένος.
	\reducevspace\reducevspace\reducevspace\reducevspace\reducevspace\reducevspace\reducevspace
	\reducevspace\reducevspace\reducevspace\reducevspace\reducevspace\reducevspace\reducevspace
	\begin{flushright}\bf{\qedsymbol{}}\end{flushright}

\end{itemize}
\end{tcolorbox}

% \hfill \break
\begin{tcolorbox}[colback=yellow!15!white, colframe=blue!50!white,
	fonttitle=\bfseries\Large, title = Απόδειξη με χρήση ιδιοτήτων κανονικών γλωσσών]
	\begin{itemize}
		\itemsep1em

	\item Οι κανονικές γλώσσες μπορεί να είναι είτε πεπερασμένες είτε μετρήσιμα άπειρες και ποτέ μη μετρήσιμα
	άπειρες. Οι πεπερασμένες γλώσσες είναι πάντα κανονικές.
	\reducevspace\reducevspace\reducevspace\reducevspace\reducevspace\reducevspace\reducevspace
	\reducevspace\reducevspace\reducevspace\reducevspace\reducevspace\reducevspace\reducevspace
	\begin{flushright}\hypertarget{4.3.8}{\bf{(8)}}\end{flushright}

	\item Οι κανονικές γλώσσες είναι κλειστές ως προς την πράξη της ένωση.
	\reducevspace\reducevspace\reducevspace\reducevspace\reducevspace\reducevspace\reducevspace
	\reducevspace\reducevspace\reducevspace\reducevspace\reducevspace\reducevspace\reducevspace
	\begin{flushright}\hypertarget{4.3.9}{\bf{(9)}}\end{flushright}

	\item $L_1 \in \mathcal{L}_{reg}$ και $L_2 \in \mathcal{L}_{reg}$ δηλαδή κανονικές.
	\reducevspace\reducevspace\reducevspace\reducevspace\reducevspace\reducevspace\reducevspace
	\reducevspace\reducevspace\reducevspace\reducevspace\reducevspace\reducevspace\reducevspace
	\begin{flushright}\hypertarget{4.3.10}{\bf{(10)}}\end{flushright}

	\item $L_1 \cup L_2 = L_3 \overset{\hyperlink{4.3.9}{\bf{\textcolor{red}{(9)}}}
	\land \hyperlink{4.3.10}{\bf{\textcolor{red}{(10)}}}}{\Longrightarrow} L_3$ κανονική
	$\overset{\hyperlink{4.3.8}{\bf{\textcolor{red}{(8)}}}}{\Longrightarrow}$ μετρήσιμη $\neq$ μή μετρήσιμη. Μόλις
	αποδείξαμε ότι οι κανονικές γλώσσες είναι κλειστές ως προς την πράξη της διαφοράς.
	\reducevspace\reducevspace\reducevspace\reducevspace\reducevspace\reducevspace\reducevspace
	\reducevspace\reducevspace\reducevspace\reducevspace\reducevspace\reducevspace\reducevspace
	\begin{flushright}\bf{\qedsymbol{}}\end{flushright}

\end{itemize}
\end{tcolorbox}

\begin{comment}
Μέθοδος 1η:

\begin{itemize}
	\itemsep0em

	\item Ως συμβολοσειρά ορίζεται ακολουθία συμβόλων πεπερασμένου μήκους. (1)

	\item Ως αλφάβητο $\Sigma$ ορίζεται ως πεπερασμένο σύνολο συμβόλων.

	\item Για πεπερασμένου πλήθους αλφάβητο $\Sigma$ και κανονική γλώσσα $\mathcal{L}$ ισχύει
  	$\mathcal{L} \subseteq \Sigma^* = \bigcup\limits_{k = 0}^{\infty}\Sigma^k$

	\item Κανονική γλώσσα απαραίτητα πρέπει να δύναται να περιγραφεί από κανονικές εκφράσεις και επίσης να
	περιγράφει κανονικά αυτόματα, τα οποία έχουν πεπερασμένη μνήμη και μπορούν να δεχθούν μόνο μετρήσιμο
	(μετρήσιμα άπειρο είτε πεπερασμένο) πλήθος, πεπερασμένου μήκους (1) συμβολοσειρές. (2)

	\item Για παράδειγμα για μια γλώσσα με αλφάβητο $\Sigma_A = {a, b}$  που περιγράφεται από την κανονική
	έκφραση $a*β*$ είναι μετρήσιμα άπειρη διότι μπορούμε μία προς μία και κατά σειρά να γράψουμε όλες τις
	(μετρήσιμα άπειρες) συμβολοσειρές που μπορεί να αυτή παράξει $\{\epsilon, a, b, ab, aa, bb,...\}$.

	\item Άρα οι κανονικές γλώσσες παράγουν μόνο πεπερασμένο ή μετρήσιμα άπειρο πλήθος συμβολοσειρών (2)

	\item Ισχύει για σύνολα Α και Β ότι $Α - Β = \{x \vert [x \in A] \land (x \notin B)\}$ και άρα
	$(A - B) \subseteq A$ (3)

\end{itemize}


Υπάρχουν κανονικές γλώσσες $\mathcal{L}_1$, $\mathcal{L}_2$. Ισχύει ότι για
$\mathcal{L}_3 = \mathcal{L}_1 - \mathcal{L}_2$, βάση του (3) αληθεύει και ότι $\mathcal{L}_3 \subseteq
\mathcal{L}_1$.

Εφόσον $\mathcal{L}_3$ υποσύνολο του $\mathcal{L}_1$ τότε δύναται να παράξει πλήθος συμβολοσειρών ίσο ή μικρότερο
και εφόσον για το $\mathcal{L}_1$ ως κανονική γλώσσα είναι είτε μετρήσιμα άπειρη ή πεπερασμένη (2) άρα το ίδιο
ισχύει και για το $\mathcal{L}_3$.


	\begin{itemize}
		\itemsep0em

		\item Ο ορισμός των συμβολοσειρών μας λέει ότι μια συμβολοσειρά είναι πάντα πεπερασμένου μήκους. (1)

		\item Οι κανονικές γλώσσες παράγουν ένα υποσύνολο συμβολοσειρών, από αυτές που δύναται να παράξει το
		αλφάβητο
		τους.

		\item Οι κανονικές γλώσσες απαραίτητα πρέπει να δύναται να περιγραφή από κανονικές εκφράσεις και επίσης να
		περιγράφουν κανονικά αυτόματα, τα οποία έχουν πεπερασμένη μνήμη και μπορούν να δεχθούν μόνο μετρήσιμο
		(μετρήσιμα άπειρο είτε πεπερασμένο) πλήθος, πεπερασμένου μήκους (1) συμβολοσειρές.

		\item Για παράδειγμα για μια γλώσσα με αλφάβητο $\Sigma_A = {a, b}$  που περιγράφεται από την κανονική
		έκφραση $a*β*$ είναι μετρήσιμα άπειρη διότι μπορούμε μία προς μία και κατά σειρά να γράψουμε όλες τις
		(μετρήσιμα άπειρες) συμβολοσειρές που μπορεί να αυτή παράξει $\{\epsilon, a, b, ab, aa, bb,...\}$.

		\item Άρα οι κανονικές γλώσσες παράγουν μόνο πεπερασμένο ή μετρήσιμα άπειρο πλήθος συμβολοσειρών (2)

		\item Ισχύει για σύνολα Α και Β ότι $Α - Β = \{x \vert [x \in A] \land (x \notin B)\}$ και άρα
		$(A - B) \subseteq A$ (3)

	\end{itemize}
\end{comment}
\begin{center}
	%\vspace{2em}
	\noindent\rule{\linewidth}{0.5pt}
	%\vspace{2em}
\end{center}
\clearpage