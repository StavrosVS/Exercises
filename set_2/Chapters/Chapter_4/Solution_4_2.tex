\subsection{Απάντηση Υποερωτήματος (β)}
\label{ssec:Solution_4.2}
\doublespacing
Ο ισχυρισμός είναι σωστός.

\begin{tcolorbox}[colback=yellow!15!white, colframe=blue!50!white,
	fonttitle=\bfseries\Large, title = Αιτιολόγηση]

Οι (ιδιότυπες) κλάσεις (σύνολα) ισοδυναμίας καθορίζονται αυστηρά και μόνο από ιδιότητες της γλώσσας.
Καθώς το πρότυπο DFA $\sim\!\!\!{M}$ είναι ελάχιστο και δίχως απρόσιτη κατάσταση, οι καταστάσεις του είναι οι ίδιες
οι κλάσεις ισοδυναμίας 1-1 προς τις κλάσεις ισοδυναμίας της $\approx\!\!{L}$. Δηλαδή και στις
δύο περιπτώσεις διαβάζοντας οποιαδήποτε συμβολοσειρά οδηγούμαστε σε ίδια κλάση ισοδυναμίας. Θα πρέπει επίσης να
θυμόμαστε ότι δεν περιγράφουν βέβαια του ίδιου τύπου δεδομένα μία και το
ένα περιγράφεται από κλάσεις ισοδυναμίας καταστάσεων ενώ το άλλο κλάσεις ισοδυναμίας συμβολοσειρών.

\end{tcolorbox}

\begin{center}
	%\vspace{2em}
	\noindent\rule{\linewidth}{0.5pt}
	%\vspace{2em}
\end{center}
\clearpage