\subsection{Απάντηση Υποερωτήματος (β)}
\label{ssec:Solution_2.2}
\doublespacing


Ο αριθμός των (εξ ορισμού ιδιότυπων) κλάσεων ισοδυναμίας αποτελεί ιδιότητα της εκάστοτε γλώσσας και είναι
ανεξάρτητος από τον αριθμό καταστάσεων οποιουδήποτε αυτόματου, παρόλα αυτά ισοδυναμεί με τον αριθμό καταστάσεων του
πρότυπου αυτόματου.
Ο αριθμός των (μη ιδιότυπων) κλάσεων ισούται με τον αριθμό των \textbf{προσβάσιμων} καταστάσεων ενός αυτόματου.
Μη πρότυπο/ελάχιστο αυτόματο ενδεχομένως να έχει παραπάνω καταστάσεις και άρα κλάσεις, που είτε να ανήκουν άνω της
μίας στην ίδια κλάση ισοδυναμίας δηλαδή ουσιαστικά είναι ισοδύναμες και άρα διπλότυπες (μη ιδιότυπες) με
οποιαδήποτε άνω της μίας να είναι περιττή, είτε να μην είναι καν προσβάσιμες (και άρα να μην ανήκουν σε καμία
κλάση, ισοδυναμίας ή μη, μια και ποτέ δεν φτάνει σε αυτές η γλώσσα που διαβάζει το αυτόματο).
Ουσιαστικά οι κλάσεις ισοδυναμίας αυτομάτου αποτελούν "συμπύκνωση" των κλάσεων/καταστάσεων του, συμψηφίζοντας αυτές
που διαβάζουν την ίδια συμβολοσειρά σε μια νέα κοινή κατάσταση/κλάση (άρα υπάρχει ελαχιστοποίηση), απορρίπτοντας
τις μη προσβάσιμες.


\begin{tcolorbox}[colback=yellow!15!white, colframe=blue!50!white,
	fonttitle=\bfseries\Large, title = Εξήγηση]
	\begin{itemize}
		\itemsep0em
		\item $\sim M$ : $\bm{5}$ κλάσεις ισοδυναμίας (αλλά 8 καταστάσεις και άρα 8 κλάσεις με προφανώς 3 να είναι
		μή ιδιότυπες και να ανήκουν σε κοινή κλάση ισοδυναμίας με κάποια/ες από τις άλλες 5 ιδιότυπες).

		\item $\sim M'$ : $\bm{5}$ ιδιότυπες κλάσεις ισοδυναμίας.

		\item $\approx L(M)$ : $\bm{5}$ ιδιότυπες κλάσεις ισοδυναμίας.

		\item $\approx L(M')$ : $\bm{5}$ ιδιότυπες κλάσεις ισοδυναμίας.
	\end{itemize}
\end{tcolorbox}



\begin{center}
	%\vspace{2em}
	\noindent\rule{\linewidth}{0.5pt}
	%\vspace{2em}
\end{center}
\clearpage
