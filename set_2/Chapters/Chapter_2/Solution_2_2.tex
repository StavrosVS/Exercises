\subsection{Απάντηση Υποερωτήματος (β)}
\label{ssec:Solution_2.2}
\doublespacing
Έχουμε να κατασκευάσουμε PDA για CFL $L_1 = \{b^n a^k b^k a^n : k, n \in  \mathbb{N}^0 \}$\\
Από τον τύπο δυνάμεθα να εξάγουμε τα εξής δεδομένα, για συμβολοσειρά $w \in L_1$:
\begin{itemize}
	\itemsep0em
	\item $\vert w \vert_{min} = 0 \Longrightarrow n,\,k = 0 \Longrightarrow w = ε $
	\item $\vert w \vert_{max} = \aleph_0$
	\item $ n + k - 2nk = 1  \Longrightarrow |w|_{min} = 2 \Longrightarrow w = ba \cup ab$
	\item $ n, k = 1 \Longrightarrow \vert w \vert_{min} = 4 \Longrightarrow w = baba$
	\item Για μεγαλύτερα $n, k$ έχουμε συμβολοσειρές όπως $bbabaa, baabba, bbaabbaa...$
	\item $ |w| = 2(n+k)$
	\item Από το παραπάνω παρατηρούμε ότι $n + k = m \in \mathbb{N_0},\,|w| = 2m\, \Longrightarrow |w|$ είναι άρτιο.
	\item Τέλος ότι $L \subseteq \{xx^R \vert x \in \Sigma^* \} $.
\end{itemize}

\par
Από αυτά κάποια δεν μας λένε κάτι ιδιαίτερα χρήσιμο, άλλα όμως μας δείχνουν την ορθή οδό που θα πρέπει να
ακολουθήσουμε. Για παράδειγμα, κατά αντιστοιχία με τις αντίστοιχες τεχνικές που ακολουθήσαμε σε αυτή τη σειρά
λύσεων στις κανονικές εκφράσεις, τα διάφορα ελάχιστα μας δίνουν την βάση πάνω στην οποία θα χτιστεί το αυτόματο
(στοίβας εν προκειμένω).
\par
Ως εκ τούτου έχουμε φαινομενικά τέσσερεις ελάχιστες περιπτώσεις, συμπεριλαμβανομένης κενής συμβολοσειράς που η
συγκεκριμένη γλώσσα επιτρέπει:
$w_0 = ε \lor w_1 = ab \lor w_2 = ba \lor w_3 = baba$ αλλά προσοχή, το $w = baba$
αποτελεί παρεμβολή του $w_1$ εντός του $w_2$ και όχι $(ba)^*$
\par
Αυτό με τη σειρά του μας λέει ότι ουσιαστικά το $w_3$ αποτελεί επέκταση της $w_2$ μετά το αρχικό (στην
πραγματικότητα αρχικά) $b$ και άρα δεν είναι βάση όλου του αυτομάτου. Παρόλα αυτά θα μας χρειαστεί ώστε να το
ολοκληρώσουμε (ουσιαστικά δείχνει ότι όταν ο αλγόριθμος θα μπαίνει στη διαδρομή που αναγιγνώσκει $ba$ θα πρέπει να
διαχωρίζεται σε δύο περιπτώσεις αυτή που μετά το $b$ είτε ακολουθεί $a$, είτε $aba$).
\par Τέλος αφού $n, k \in \mathbb{N_0}$ τότε μπορούν όλοι αυτοί οι χαρακτήρες να είναι (μετρήσιμα) άπειροι αλλά υπό
τον όρο ότι τηρείται το γενικό μοτίβο.
\par Άρα, θεωρητικά, έχουμε 3 αρχικούς κλάδους και  και σε έναν από αυτούς άλλη μία διάσπαση σε δύο. Ξεκινάμε την
κατασκευή:


\hfill \break
\par
Είμαστε έτοιμοι να συνθέσουμε το τελικό αυτόματο $M_{\text{Τ}}$ με $Q_{\text{Τ}} = Q_{1} \bm{\times} Q_{2}
=$\\ $\{(q_1,\, q_2) \;\vert\; q_1 \in Q_1 \in M1, q_2 \in Q_2 \in M2\}$, όθεν επίσης δείχνει ότι έχουμε\\ $\vert
Q_{1}\vert \bm{\cdot} \vert Q_{2}\vert = \vert
Q_{\text{Τ}}\vert_{max} \Longrightarrow 2\bm{\cdot} 4 = 8$:
\clearpage

Κόμβοι:
\reducevspace\reducevspace\reducevspace\reducevspace\reducevspace\reducevspace\reducevspace
\begin{multicols}{4}
\begin{itemize}
	\itemsep0em
	\item $q_T^{0} = (q_1^{0}, q_2^{0})$
	\item $q_T^{1} = (q_1^{0}, q_2^{1})$

	\item $q_T^{2} = (q_1^{0}, q_2^{2})$
	\item $q_T^{3} = (q_1^{0}, q_2^{3})$

	\item $q_T^{4} = (q_1^{1}, q_2^{0})$
	\item $q_T^{5} = (q_1^{1}, q_2^{1})$

	\item $q_T^{6} = (q_1^{1}, q_2^{2})$
	\item $q_T^{7} = (q_1^{1}, q_2^{3})$
\end{itemize}
\end{multicols}
\reducevspace\reducevspace\reducevspace\reducevspace\reducevspace\reducevspace\reducevspace
\par Μεταβάσεις
\reducevspace\reducevspace\reducevspace\reducevspace\reducevspace\reducevspace\reducevspace
\begin{multicols}{2}
\begin{itemize}
	\itemsep0em
	\item $q_T^{0} = (q_1^{0}, q_2^{0}) \bm{\rightarrow^{a}} (q_1^{1}, q_2^{0}) = q_T^{4}$
	\item $q_T^{1} = (q_1^{0}, q_2^{1}) \bm{\rightarrow^{a}} (q_1^{1}, q_2^{1}) = q_T^{5}$
	\item $q_T^{2} = (q_1^{0}, q_2^{2}) \bm{\rightarrow^{a}} (q_1^{1}, q_2^{2}) = q_T^{6}$
	\item $q_T^{3} = (q_1^{0}, q_2^{3}) \bm{\rightarrow^{a}} (q_1^{1}, q_2^{3}) = q_T^{7}$ %%%%
	\item $q_T^{4} = (q_1^{1}, q_2^{0}) \bm{\rightarrow^{a}} (q_1^{0}, q_2^{0}) = q_T^{0}$
	\item $q_T^{5} = (q_1^{1}, q_2^{1}) \bm{\rightarrow^{a}} (q_1^{0}, q_2^{1}) = q_T^{1}$
	\item $q_T^{6} = (q_1^{1}, q_2^{2}) \bm{\rightarrow^{a}} (q_1^{0}, q_2^{2}) = q_T^{2}$
	\item $q_T^{7} = (q_1^{1}, q_2^{3}) \bm{\rightarrow^{a}} (q_1^{0}, q_2^{3}) = q_T^{3}$ %%%%

	\item $q_T^{0} = (q_1^{0}, q_2^{0}) \bm{\rightarrow^{b}} (q_1^{0}, q_2^{1}) = q_T^{1}$
	\item $q_T^{1} = (q_1^{0}, q_2^{1}) \bm{\rightarrow^{b}} (q_1^{0}, q_2^{2}) = q_T^{2}$
	\item $q_T^{2} = (q_1^{0}, q_2^{2}) \bm{\rightarrow^{b}} (q_1^{0}, q_2^{3}) = q_T^{3}$
	\item $q_T^{3} = (q_1^{0}, q_2^{3}) \bm{\rightarrow^{b}} (q_1^{0}, q_2^{0}) = q_T^{0}$ %%%%
	\item $q_T^{4} = (q_1^{1}, q_2^{0}) \bm{\rightarrow^{b}} (q_1^{1}, q_2^{1}) = q_T^{5}$
	\item $q_T^{5} = (q_1^{1}, q_2^{1}) \bm{\rightarrow^{b}} (q_1^{1}, q_2^{2}) = q_T^{6}$
	\item $q_T^{6} = (q_1^{1}, q_2^{2}) \bm{\rightarrow^{b}} (q_1^{1}, q_2^{3}) = q_T^{7}$
	\item $q_T^{7} = (q_1^{1}, q_2^{3}) \bm{\rightarrow^{b}} (q_1^{1}, q_2^{0}) = q_T^{4}$ %%%%
\end{itemize}
\end{multicols}
\reducevspace\reducevspace\reducevspace\reducevspace\reducevspace\reducevspace\reducevspace
\par
Έχουμε DFA $M_{\text{Τ}}$ με εξής μη ενδιαμέσους κόμβους:\\
αρχικό $s_{\text{Τ}} \in M_{\text{Τ}} = \{(q_1,\, q_2) \;\vert\; q_1 = s_1 \in M_1, q_2 = s_2 \in M_2 \}
\Longrightarrow q_{\text{Τ}}^0 = {q_1^0, q_2^0}$,\\
τελικό $F_{\text{Τ}}\in M_{\text{Τ}} =\{(q_1,\, q_2) \;\vert\; q_1\in F_1 \in M_1, q_2\in F_2\in M_2 \}
\Longrightarrow q_{\text{Τ}}^3 = {q_1^0, q_2^3}$\\
\reducevspace\reducevspace\reducevspace\reducevspace\reducevspace\reducevspace\reducevspace
\hfill \break
DFA $ M_{\text{Τ}} = (\; \{\,q0,\, q1,\, q2,\, q3,\, q4,\, q5,\,q_6,\,q_7\,\},\; \{\,a,\, b\,\},\\
\{\; δ(q0,\; a)=q4,\; δ(q0, b)=q1,\; δ(q1, a)=q5,\; δ(q1, b)=q2,\\
δ(q2, a)=q6,\; δ(q2, b)=q3,\; δ(q3, a)=q7,\; δ(q3, b)=q0,\\
δ(q4, a)=q0,\; δ(q4, b)=q5,\; δ(q5, a)=q1,\; δ(q5, b)=q6,\\
δ(q6, a)=q2,\; δ(q6, b)=q7,\; δ(q7, a)=q3,\; δ(q7, b)=q4\;\},\\
q0,\; \{\,q3\,\}\;)$

%\par
\begin{comment}
\begin{figure}[!htb]
	\centering
	%\begin{minipage}{0.45\textwidth}
	%\begin{flushleft}
	\begin{tikzpicture} [blue, node distance = 2.5cm, on grid, font=\sffamily\large\bfseries]
		% Help grid
		%	\draw [help lines] (-1,4) grid (8,-10);
		% Start Node : Every other node is measured based to this one
		\node (q0) [state, black, scale = 1.3, initial below, initial distance = 10mm, initial text = {Αρχή},
		top color = green!80, bottom color = green!20] {$q_{0}$};

		% Final Node
		\node (q3) [state, black, scale = 1.3,accepting, top color = red!80, bottom color = red!20, double
		distance = 3pt, double = red!20, left = of q0] {$q_{3}$};

		\node (q1) [state, black, scale = 1.3, top color = gray!80, bottom color = gray!20, right = of q0] {$q_{1}$};

		\node (q2) [state, black, scale = 1.3, top color = gray!80, bottom color = gray!20, right = of q1] {$q_{2}$};

		\node (q4) [state, black, scale = 1.3, top color = gray!80, bottom color = gray!20, above = of q0] {$q_{4}$};

		\node (q5) [state, black, scale = 1.3, top color = gray!80, bottom color = gray!20, right = of q4] {$q_{5}$};

		\node (q6) [state, black, scale = 1.3, top color = gray!80, bottom color = gray!20, right = of q5] {$q_{6}$};

		\node (q7) [state, black, scale = 1.3, top color = gray!80, bottom color = gray!20, left = of q4] {$q_{7}$};

		% Draw Arrows/Connections
		\path [-stealth, thick]

		% 0,0 => 1,0 || 0,1
		(q0) edge [bend left] node [left] {$a$}(q4)
		(q0) edge [] node [below] {$b$} (q1)
		% 0,1 => 1,1 || 0,2
		(q1) edge [bend left] node [left] {$a$} (q5)
		(q1) edge [] node [below] {$b$} (q2)
		% 0,2 => 1,2 || 0,3
		(q2) edge [bend left] node [left] {$a$} (q6)
		(q2) edge [bend left] node [above] {$b$} (q3)
		% 0,3 => 1,3 || 0,0
		(q3) edge [bend left] node [left] {$a$} (q7)
		(q3) edge [] node [below] {$b$} (q0)
		% 1,0 => 0,0 || 1,1
		(q4) edge [bend left] node [right] {$a$} (q0)
		(q4) edge [] node [above] {$b$} (q5)
		% 1,1 => 0,1 || 1,2
		(q5) edge [bend left] node [right] {$a$} (q1)
		(q5) edge [] node [above] {$b$} (q6)
		% 1,2 => 0,2 || 1,3
		(q6) edge [bend left] node [right] {$a$} (q2)
		(q6) edge [bend right] node [below] {$b$} (q7)
		% 1,3 => 0,3 || 1,0
		(q7) edge [bend left] node [right] {$a$} (q3)
		(q7) edge [] node [above] {$b$} (q4);
	\end{tikzpicture}
	\caption{DFA $M_{1}$ : \\ $L = \{\,w\; \vert \; (w \in \{a, b\}^*) \; [\vert w \vert$ περιττό$]\}$}
	\label{fig:sub1}
\end{figure}
\end{comment}

\vfill
\clearpage
\begin{tcolorbox}[colback=yellow!15!white, colframe=blue!50!white,
	fonttitle=\bfseries\Large, title = DFA $M_{\text{τελικό}}$]
	\centering
	\begin{tikzpicture} [blue, node distance = 2.5cm, on grid, font=\sffamily\large\bfseries]
		% Help grid
		%	\draw [help lines] (-1,4) grid (8,-10);
		% Start Node : Every other node is measured based to this one
		\node (q0) [state, black, scale = 1.3, initial below, initial distance = 10mm, initial text = {Αρχή},
		top color = green!80, bottom color = green!20] {$q_{0}$};

		% Final Node
		\node (q3) [state, black, scale = 1.3,accepting, top color = red!80, bottom color = red!20, double
		distance = 3pt, double = red!20, left = of q0] {$q_{3}$};

		\node (q1) [state, black, scale = 1.3, top color = gray!80, bottom color = gray!20, right = of q0]
		{$q_{1}$};

		\node (q2) [state, black, scale = 1.3, top color = gray!80, bottom color = gray!20, right = of q1]
		{$q_{2}$};

		\node (q4) [state, black, scale = 1.3, top color = gray!80, bottom color = gray!20, above = of q0]
		{$q_{4}$};

		\node (q5) [state, black, scale = 1.3, top color = gray!80, bottom color = gray!20, right = of q4]
		{$q_{5}$};

		\node (q6) [state, black, scale = 1.3, top color = gray!80, bottom color = gray!20, right = of q5]
		{$q_{6}$};

		\node (q7) [state, black, scale = 1.3, top color = gray!80, bottom color = gray!20, left = of q4] {$q_{7}$};

		% Draw Arrows/Connections
		\path [-stealth, thick]

		% 0,0 => 1,0 || 0,1
		(q0) edge [bend left] node [left] {$a$}(q4)
		(q0) edge [] node [below] {$b$} (q1)
		% 0,1 => 1,1 || 0,2
		(q1) edge [bend left] node [left] {$a$} (q5)
		(q1) edge [] node [below] {$b$} (q2)
		% 0,2 => 1,2 || 0,3
		(q2) edge [bend left] node [left] {$a$} (q6)
		(q2) edge [bend left] node [above] {$b$} (q3)
		% 0,3 => 1,3 || 0,0
		(q3) edge [bend left] node [left] {$a$} (q7)
		(q3) edge [] node [below] {$b$} (q0)
		% 1,0 => 0,0 || 1,1
		(q4) edge [bend left] node [right] {$a$} (q0)
		(q4) edge [] node [above] {$b$} (q5)
		% 1,1 => 0,1 || 1,2
		(q5) edge [bend left] node [right] {$a$} (q1)
		(q5) edge [] node [above] {$b$} (q6)
		% 1,2 => 0,2 || 1,3
		(q6) edge [bend left] node [right] {$a$} (q2)
		(q6) edge [bend right] node [below] {$b$} (q7)
		% 1,3 => 0,3 || 1,0
		(q7) edge [bend left] node [right] {$a$} (q3)
		(q7) edge [] node [above] {$b$} (q4);
	\end{tikzpicture}
\end{tcolorbox}
\begin{center}
	%\vspace{2em}
	\noindent\rule{\linewidth}{0.5pt}
	%\vspace{2em}
\end{center}
%\clearpage
