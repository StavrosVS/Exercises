\subsection{Απάντηση Υποερωτήματος (β)}
\label{ssec:Solution_2.2}
\doublespacing
Έχουμε να κατασκευάσουμε PDA για CFL $L_2 = \{w \in \{a,\,b\}^* : |w|_b = 2|w|_a \}$\\
Από τον τύπο δυνάμεθα να εξάγουμε τα εξής δεδομένα, για συμβολοσειρά $w \in L_1$:
\reducevspace\reducevspace\reducevspace\reducevspace\reducevspace\reducevspace
\begin{itemize}
	\itemsep0em
	\item $\vert w \vert_{min} = 0 \Longrightarrow w_{min} = ε$, \quad $\vert w \vert_{max} =
	\aleph_0$, \quad $|w| \equiv |w|_a \pmod{2}$
	\item $\{w \in \{a,\,b\}^+ \;\vert\; |w| = 3,\, |w|_a = 1,\, |w|_b = 2\} \Longrightarrow |w| = 3|w|_a$
	\item $|w| = 3\,:\, abb, bab, bba$
	\item $|w| > 3\,:\, aabbbb,\, bbbbaa,\, abbbba,\, babbab,\, ababbb,\, bbbaba,\, bbaabb,\,...$
	\item Κανένας περιορισμός οποιουδήποτε συμβόλου ως προς την θέση.
\end{itemize}
\reducevspace\reducevspace
\par
Έχουμε φαινομενικά τέσσερεις ελάχιστες περιπτώσεις, που η συγκεκριμένη γλώσσα επιτρέπει:
$w_0 = ε \lor w_1 = abb \lor w_2 = bab \lor w_3 = bba$ αλλά όπως ήδη είπαμε δεν υπάρχει περιορισμός ως προς την
θέση αντίθετα με το προηγούμενο πρόβλημα στην άσκηση 2.1 (υποερώτημα (α)).
\par
Αυτό με τη σειρά του δείχνει ότι πιθανώς να μην χρειαστεί κάποια διακλάδωση και να μπορούν να γίνουν όλες οι
διαδικασίες αναδρομικά στον ίδιο κόμβο.
\par Άρα, υποθέτουμε ένα κόμβο για όλη την ανάγνωση βασιζόμενοι σε σύνθετες αναδρομές, συν επιπρόσθετους
βοηθητικούς κόμβους (Εναρκτήριο με μετάβαση αρχικοποίησης σωρού με δείκτη τελευταίου κελιού και τελικό με μετάβαση
προς αυτόν όπου αδειάζει τον δείκτη από τον σωρό). Ξεκινάμε την κατασκευή του αυτομάτου, αρχικά δίνοντας την
μαθηματική περιγραφή και κατόπιν το αντίστοιχο διάγραμμα:

\begin{tcolorbox}[colback=yellow!15!white, colframe=blue!50!white,
	fonttitle=\bfseries\Large, title = {PDA $M = (K,\, \Sigma,\, \Gamma,\, \Delta,\, s,\, F)$}]
\begin{itemize}
	\itemsep0em
	\item $K \,=\, \{q1,\, q2,\, q3\}$
	\reducevspace\reducevspace\reducevspace
	\item $\Sigma \,=\, \{a,\, b\}$
	\reducevspace\reducevspace\reducevspace
	\item $\Gamma \,=\, \{A,\, B,\, \$\}$
	\reducevspace\reducevspace\reducevspace
	\item $\Delta \,=\,$
	\reducevspace\reducevspace\reducevspace\reducevspace\reducevspace\reducevspace\reducevspace
	\reducevspace\reducevspace\reducevspace\reducevspace\reducevspace\reducevspace\reducevspace
		\begin{multicols}{2}
		\begin{enumerate}
			\item $(q1,\,ε,\,ε)\rightarrow(q2,\,\$)$



			\item $(q2,\,b,\,\$)\rightarrow(q2,\,B\$)$
			\item $(q2,\,b,\,B)\rightarrow(q2,\,BB)$
			\item $(q2,\,b,\,Α)\rightarrow(q2,\,ε)$

			\item $(q2,\,a,\,\$)\rightarrow(q2,\,AA\$)$
			\item $(q2,\,a,\,A)\rightarrow(q2,\,AAA)$
			\item $(q2,\,a,\,BB)\rightarrow(q2,\,ε)$
			\item $(q2,\,a,\,B\$)\rightarrow(q2,\,A\$)$

			\item $(q2,\,ε,\,\$)\rightarrow(q3,\,ε)$
		\end{enumerate}
		\end{multicols}
		\reducevspace\reducevspace\reducevspace\reducevspace\reducevspace
		\reducevspace\reducevspace\reducevspace\reducevspace\reducevspace
	\item $s \,=\, q_1$
	\reducevspace\reducevspace\reducevspace
	\item $F \,=\, q_3$
\end{itemize}
\end{tcolorbox}
\reducevspace\reducevspace\reducevspace\reducevspace

%\par
\begin{comment}
\begin{figure}[!htb]
	\centering
	%\begin{minipage}{0.45\textwidth}
	%\begin{flushleft}
	\begin{tikzpicture} [blue, node distance = 2.5cm, on grid, font=\sffamily\large\bfseries]
		% Help grid
		%	\draw [help lines] (-1,4) grid (8,-10);
		% Start Node : Every other node is measured based to this one
		\node (q0) [state, black, scale = 1.3, initial below, initial distance = 10mm, initial text = {Αρχή},
		top color = green!80, bottom color = green!20] {$q_{0}$};

		% Final Node
		\node (q3) [state, black, scale = 1.3,accepting, top color = red!80, bottom color = red!20, double
		distance = 3pt, double = red!20, left = of q0] {$q_{3}$};

		\node (q1) [state, black, scale = 1.3, top color = gray!80, bottom color = gray!20, right = of q0] {$q_{1}$};

		\node (q2) [state, black, scale = 1.3, top color = gray!80, bottom color = gray!20, right = of q1] {$q_{2}$};

		\node (q4) [state, black, scale = 1.3, top color = gray!80, bottom color = gray!20, above = of q0] {$q_{4}$};

		\node (q5) [state, black, scale = 1.3, top color = gray!80, bottom color = gray!20, right = of q4] {$q_{5}$};

		\node (q6) [state, black, scale = 1.3, top color = gray!80, bottom color = gray!20, right = of q5] {$q_{6}$};

		\node (q7) [state, black, scale = 1.3, top color = gray!80, bottom color = gray!20, left = of q4] {$q_{7}$};

		% Draw Arrows/Connections
		\path [-stealth, thick]

		% 0,0 => 1,0 || 0,1
		(q0) edge [bend left] node [left] {$a$}(q4)
		(q0) edge [] node [below] {$b$} (q1)
		% 0,1 => 1,1 || 0,2
		(q1) edge [bend left] node [left] {$a$} (q5)
		(q1) edge [] node [below] {$b$} (q2)
		% 0,2 => 1,2 || 0,3
		(q2) edge [bend left] node [left] {$a$} (q6)
		(q2) edge [bend left] node [above] {$b$} (q3)
		% 0,3 => 1,3 || 0,0
		(q3) edge [bend left] node [left] {$a$} (q7)
		(q3) edge [] node [below] {$b$} (q0)
		% 1,0 => 0,0 || 1,1
		(q4) edge [bend left] node [right] {$a$} (q0)
		(q4) edge [] node [above] {$b$} (q5)
		% 1,1 => 0,1 || 1,2
		(q5) edge [bend left] node [right] {$a$} (q1)
		(q5) edge [] node [above] {$b$} (q6)
		% 1,2 => 0,2 || 1,3
		(q6) edge [bend left] node [right] {$a$} (q2)
		(q6) edge [bend right] node [below] {$b$} (q7)
		% 1,3 => 0,3 || 1,0
		(q7) edge [bend left] node [right] {$a$} (q3)
		(q7) edge [] node [above] {$b$} (q4);
	\end{tikzpicture}
	\caption{DFA $M_{1}$ : \\ $L = \{\,w\; \vert \; (w \in \{a, b\}^*) \; [\vert w \vert$ περιττό$]\}$}
	\label{fig:sub1}
\end{figure}
\end{comment}

%\vfill
%\clearpage
\begin{tcolorbox}[colback=yellow!15!white, colframe=blue!50!white,
	fonttitle=\bfseries\Large, title = DFA $M_{\text{τελικό}}$]
	\reducevspace\reducevspace\reducevspace\reducevspace\reducevspace\reducevspace
	\reducevspace\reducevspace\reducevspace\reducevspace\reducevspace\reducevspace
	\reducevspace\reducevspace\reducevspace\reducevspace\reducevspace\reducevspace
	\centering
	\begin{tikzpicture} [blue, node distance = 4cm, on grid, font=\sffamily\large\bfseries]
		% Help grid
		%	\draw [help lines] (-1,4) grid (8,-10);
		% Start Node : Every other node is measured based to this one
		\node (q1) [state, black, scale = 1.3, initial left, initial distance = 10mm, initial text = {Αρχή},
		top color = green!80, bottom color = green!20] {$q_{1}$};

		\node (q2) [state, black, scale = 1.3, top color = gray!80, bottom color = gray!20, right = of q1]
		{$q_{2}$};

		% Final Node
		\node (q3) [state, black, scale = 1.3,accepting, top color = red!80, bottom color = red!20, double
		distance = 3pt, double = red!20, right = of q2] {$q_{3}$};

		% Draw Arrows/Connections
		\path [-stealth, thick]

		% 0,0 => 1,0 || 0,1
		(q1) edge [] node [above] {$ε,\,ε,\,\$$} (q2)
		% 0,1 => 1,1 || 0,2
		(q2) edge [loop above] node [pos=0.5, yshift=-10pt, inner sep=1pt]
			{$\begin{array}{c}
				a,\$,AA\$ \\[-30pt]
				a,A,AAA \\[-30pt]
				a,B\$,A\$ \\[-30pt]
				a,BB,ε
			\end{array}$} (q2)
		(q2) edge [loop below] node [pos=0.5, yshift=20pt, inner sep=1pt]
		{$\begin{array}{c}
				b,\$,B\$ \\[-30pt]
				b,A,ε \\[-30pt]
				b,B,BB
			\end{array}$} (q2)

		(q2) edge [] node [above] {$ε,\$,ε$} (q3);
	\end{tikzpicture}
	\reducevspace\reducevspace\reducevspace\reducevspace\reducevspace\reducevspace
	\reducevspace\reducevspace\reducevspace\reducevspace\reducevspace\reducevspace
\end{tcolorbox}
\begin{center}
	%\vspace{2em}
	\noindent\rule{\linewidth}{0.5pt}
	%\vspace{2em}
\end{center}
%\clearpage
