\subsection{Απάντηση Υποερωτήματος (α)}
\label{ssec:Solution_1.1}
\doublespacing
Έχουμε CFL
$ L_1 = \{a^n, b^m : n, m \in \mathbb{N}_0, 3n \leq m \leq 6n \}$
άρα $\vert w \vert_{min} = 0$ για $n = 0$ με $a$ να προηγούνται των $b$. Οπότε:
$\overset{\text{Ελάχιστη περίπτωση}}{\rightarrow} ε\,$ και
$\;\,\overset{\text{Ελάχιστη μή κενή περίπτωση}}{\rightarrow} abbb$.\\
Αλλά με το ίδιο $a$ έχουμε επίσης: $\rightarrow a\,bbb\,b,\; a\,bbb\,bb,\; a\,bbb\,bbb$.\\
Δεν πρέπει όμως να ξεχνάμε ότι μπορούμε να έχουμε έως $|w|_a = \aleph_0$ και αντίστοιχα τα $b$ που αναλογούν σε
αυτά για να είναι σωστή η συμβολοσειρά. Οτιδήποτε όμως από αυτά είναι ουσιαστικά απλή επανάληψη του κανόνα που
δημιουργεί όλες τις ενός $a$ συμβολοσειρές. Άρα δεν χρειαζόμαστε κάποιο
επιπρόσθετο κανόνα παραγωγής αλλά έχουμε αναδρομή.\\
Συμψηφίζοντας τώρα αυτές τις παρατηρήσεις μπορούμε να περάσουμε στην κατασκευή των κανόνων παραγωγής. Θα το
προσεγγίσουμε σταδιακά:
\begin{itemize}
	\itemsep0em

	\item Ελάχιστη περίπτωση:\\
	$S \rightarrow ε$

	\item Ελάχιστα $a$:\\
	$S \rightarrow aΜ \,|\, ε$\\
	$M \rightarrow bbb \,|\, bbbb \,|\, bbbbb \,|\, bbbbbb$

	\item $|w| = \aleph_0$:\\
	$S \rightarrow aSM \,|\, ε$\\
	$M \rightarrow bbb \,|\, bbbb \,|\, bbbbb \,|\, bbbbbb$

	\item Πιο ευανάγνωστη παραλλαγή:\\
	$S \rightarrow aSbbM | ε$\\
	$M \rightarrow b \,|\, bb \,|\, bbb \,|\, bbbb$

	\item Καμία αριστερή αναδρομή.
\end{itemize}

\begin{tcolorbox}[colback=yellow!15!white, colframe=blue!50!white,
	fonttitle=\bfseries\Large, title = Γραμματική και συντακτικό δένδρο]
Είμαστε έτοιμοι να προχωρήσουμε στην πλήρη περιγραφή της γραμματικής μας:\\

$G_1 = (\{a, b\},\, \{ε\},\, \{S\rightarrow aSbbM\,|\,ε,\; Μ\rightarrow b \;|\; bb \;|\; bbb \;|\; bbbb\},\, S)$

Ήρθε η στιγμή να δώσουμε το συντακτικό δένδρο για συμβολοσειρά (τα κενά για ευκολότερη ανάγνωση) $w =
aa\,bbb\,bbb\,b,\,|w| = 9,\, |w|_a = 2,\,|w|_b = 7$:


\begin{center}
	\Tree
	[.S
		[.$a$ ]
		[.{S}
			[.$a$ ]
			[.{S}
				[.$ε$ ]
			]
			[.{$bb$} ]
			[.{M}
				[.$b$ ]
			]
		]
		[.{$bb$} ]
		[.{M}
			[.{$bb$} ]
		]
	]
\end{center}

Που δίνει:$\qquad\qquad\qquad\;\;\, a\;\;\;\, a\qquad\; bb\;\;\;\, b\;\;\;\, bb\;\;\; bb$\\
Εναλλακτικά θα μπορούσαμε να το διαβάσουμε ως $a\,(a(\varepsilon)bb(bb))\,bb\,(b)$.

\end{tcolorbox}



\begin{center}
	%\vspace{2em}
	\noindent\rule{\linewidth}{0.5pt}
	%\vspace{2em}
\end{center}
%\clearpage