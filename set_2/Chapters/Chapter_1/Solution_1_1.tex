\subsection{Απάντηση Υποερωτήματος (α)}
\label{ssec:Solution_1.1}
\doublespacing
Έχουμε CFL
$ L_1 = \{a^n, b^m : n, m \in \mathbb{N}_0, 3n \leq m \leq 6n \}$
άρα $\vert w \vert_{min} = 0$ για $n = 0$ με $a$ να προηγούνται των $b$. Οπότε:
$\overset{\text{Ελάχιστη περίπτωση}}{\rightarrow} ε\,$ και
$\;\,\overset{\text{Ελάχιστη μή κενή περίπτωση}}{\rightarrow} abbb$.\\
Αλλά με το ίδιο $a$ έχουμε επίσης (με κενά για να είναι ευανάγνωστο): $\rightarrow a\,bbb\,b,\;
a\,bbb\,bb, \; a\,bbb\,bbb$.\\
Δεν πρέπει όμως να ξεχνάμε ότι μπορούμε να έχουμε έως $|w|_a = \aleph_0$ και αντίστοιχα τα $b$ που αναλογούν σε
αυτά για να είναι σωστή η συμβολοσειρά. Οτιδήποτε όμως από αυτά είναι ουσιαστικά απλή επανάληψη του κανόνα που
δημιουργεί όλες τις ενός $a$ συμβολοσειρές. Άρα δεν χρειαζόμαστε κάποιο
επιπρόσθετο κανόνα παραγωγής αλλά έχουμε αναδρομή.\\
Συμψηφίζοντας τώρα αυτές τις παρατηρήσεις μπορούμε να περάσουμε στην κατασκευή των κανόνων παραγωγής. Θα το
προσεγγίσουμε σταδιακά:
\begin{itemize}
	\itemsep0em

	\item Ελάχιστη περίπτωση:\\
	$S \rightarrow ε$

	\item Ελάχιστα $a$:\\
	$S \rightarrow aΜ \,|\, ε$\\
	$M \rightarrow bbb \,|\, bbbb \,|\, bbbbb \,|\, bbbbbb$

	\item $|w| = \aleph_0$:\\
	$S \rightarrow aSM \,|\, ε$\\
	$M \rightarrow bbb \,|\, bbbb \,|\, bbbbb \,|\, bbbbbb$

	\item Πιο ευανάγνωστη παραλλαγή:\\
	$S \rightarrow aSbbM | ε$\\
	$M \rightarrow b \,|\, bb \,|\, bbb \,|\, bbbb$
\end{itemize}

\begin{tcolorbox}[colback=yellow!15!white, colframe=blue!50!white,
	fonttitle=\bfseries\Large, title = Γραμματική και συντακτικό δένδρο]
Είμαστε έτοιμοι να προχωρήσουμε στην πλήρη περιγραφή της γραμματικής μας:\\

$G_1 = (\{a, b\},\, \{ε\},\, \{S\rightarrow aSbbM\,|\,ε,\; Μ\rightarrow b \;|\; bb \;|\; bbb \;|\; bbbb\},\, S)$

Ήρθε η στιγμή να δώσουμε το συντακτικό δένδρο για συμβολοσειρά (τα κενά για ευκολότερη ανάγνωση) $w =
aa\,bbb\,bbb\,b,\,|w| = 9,\, |w|_a = 2,\,|w|_b = 7$:


\begin{center}
	\Tree
	[.S
		[.$a$ ]
		[.{S}
			[.$a$ ]
			[.{S}
				[.$ε$ ]
			]
			[.{$bb$} ]
			[.{M}
				[.$b$ ]
			]
		]
		[.{$bb$} ]
		[.{M}
			[.{$bb$} ]
		]
	]
\end{center}

Που δίνει:$\qquad\qquad\qquad\;\;\, a\;\;\;\, a\qquad\; bb\;\;\;\, b\;\;\;\, bb\;\;\; bb$\\
Εναλλακτικά θα μπορούσαμε να ανταλλάξουμε το ένα $\,Μ\,$ με το άλλο και θα φτάναμε πάλι στο ίδιο αποτέλεσμα.

\end{tcolorbox}

\begin{comment}
\reducevspace
%\phantom{\text{Ελάχιστη περίπτωση}}
\begin{equation*}
	\overset{\text{Ελάχιστη περίπτωση }}{\rightarrow} ab \cup ba
	\reducevspace\reducevspace\reducevspace\reducevspace\reducevspace\reducevspace\reducevspace
	\reducevspace\reducevspace\reducevspace\reducevspace\reducevspace\reducevspace\reducevspace
\end{equation*}
\begin{equation*}
	\overset{\vert w \vert_c = 2n }{\rightarrow} (cc)^* [a(cc)^*b \cup b(cc)^*a](cc)^*
\end{equation*}
\begin{equation*}
	\begin{split}
		&\overset{\text{cacb, acbc, cbca, cabc, cccabc\dots}}{\rightarrow}(cc)^*[(a[cc]^*b) \cup (b[cc]^*a) \cup
		(cac[cc]^*b) \cup\\
		&(cbc[cc]^*a) \cup (ca[cc]^*bc) \cup (cb[cc]^*ac) \cup (ac[cc]^*bc) \cup
		(bc[cc]^*ac)](cc)^*
	\end{split}
\end{equation*}
\par
Θα τη "σπάσουμε" σε τμήματα χάριν ευκολίας ανάγνωσης:


\begin{comment}
\phantom{\text{Τελική κανονική έκφραση χωρισμένη βάση μοτίβου}}
\begin{equation*}
	{\boldmath
	\begin{split}
		\begin{gathered}
			(cc)^*\textcolor{purple}{[}
			\\ (a[cc]^*b) \cup (b[cc]^*a)
			\\ \cup
			\\ (cac[cc]^*b) \cup (cbc[cc]^*a)
			\\ \cup
			\\ (ca[cc]^*bc) \cup (cb[cc]^*ac)
			\\ \cup
			\\ (ac[cc]^*bc) \cup (bc[cc]^*ac)
			\\ \textcolor{purple}{]}(cc)^*
		\end{gathered}
	\end{split}}
	\;\;\;\;\;\;\;\;\;\;\text{Τελική κανονική έκφραση χωρισμένη βάση μοτίβου}
\end{equation*}




\reducevspace\reducevspace\reducevspace\reducevspace\reducevspace\reducevspace
\mathversion{bold}
\begin{tcolorbox}[colback=yellow!15!white, colframe=blue!50!white,
	fonttitle=\bfseries\Large, title = Τελική κανονική έκφραση χωρισμένη βάση μοτίβου]
	\reducevspace\reducevspace\reducevspace\reducevspace\reducevspace\reducevspace\reducevspace
	\reducevspace\reducevspace\reducevspace\reducevspace\reducevspace\reducevspace\reducevspace
	\reducevspace\reducevspace\reducevspace\reducevspace\reducevspace\reducevspace\reducevspace
	\begin{align*}
		\begin{gathered}
		(cc)^*(
		\\ \textcolor{red}{[}\textcolor{blue}{(}a\,[cc]^*\,b\textcolor{blue}{)} \;\;\textcolor{blue}{\cup}\;\;
		\textcolor{blue}{(}b\,[cc]^*\,a\textcolor{blue}{)}\textcolor{red}{]}
		\\ \textcolor{red}{\cup}
		\\ \textcolor{red}{[}\textcolor{blue}{(}cac\,[cc]^*\,b\textcolor{blue}{)} \;\;\textcolor{blue}{\cup}\;\;
		\textcolor{blue}{(}cbc\,[cc]^*\,a\textcolor{blue}{)}\textcolor{red}{]}
		\\ \textcolor{red}{\cup}
		\\ \textcolor{red}{[}\textcolor{blue}{(}ca\,[cc]^*\,bc\textcolor{blue}{)} \;\;\textcolor{blue}{\cup}\;\;
		\textcolor{blue}{(}cb\,[cc]^*\,ac\textcolor{blue}{)}\textcolor{red}{]}
		\\ \textcolor{red}{\cup}
		\\ \textcolor{red}{[}\textcolor{blue}{(}ac\,[cc]^*\,bc\textcolor{blue}{)} \;\;\textcolor{blue}{\cup}\;\;
		\textcolor{blue}{(}bc\,[cc]^*\,ac\textcolor{blue}{)}\textcolor{red}{]}
		\\ )(cc)^*
		\end{gathered}
	\end{align*}
\end{tcolorbox}
\mathversion{normal}
\end{comment}
%Έχουμε περιττό πλήθος "$a$" άρα:
%\[ \{m(a) : \exists n(n \in \mathbb N) \land \ m(a) = 2n+1\} \]
%Αυτό μας δείχνει ότι έχουμε τουλάχιστον ένα $a$ σε κάθε
%περίπτωση δηλαδή την λέξη:
%\[w = a\]
%Επιπλέον έχουμε ακριβώς δύο "$b$". Άρα οι απλούστερες
%περιπτώσεις όπου μας δίνουν τις αντίστοιχες κανονικές
%εκφράσεις
%ως βάση πάνω στην οποία θα δουλέψουμε είναι:
%\[ w_{1_a} = abb,\qquad w_{1_b} = bab,\qquad w_{1_c} = bba \]
%Επειδή δεν έχουμε περιορισμό στο πλήθος "$a$", αρκεί αυτό να
%είναι περιττό, θα πρέπει να παρέχουμε αυτή τη δυνατότητα. Μόνο
%προσθέτοντας άρτιο σε περιττό μας δίνει περιττό:
%\[ m(a) = (2n+1)+(2m) \Leftrightarrow 2n+2m+1 \Leftrightarrow
%2(n+m)+1 \Leftrightarrow 2k+1\]
%\[\{(n, m) \in \mathbb N : k = n+m \ \land \ (2 | n \ \lor \ n
%= 0) \ \land \ 2 \centernot| m\} \]
%\[ w_{2_a} = a(aa)^{*}b(aa)^{*}b(aa)^{*},\quad w_{2_b} =
%(aa)^{*}ba(aa)^{*}b(aa)^{*},\quad w_{2_c} =
%(aa)^{*}b(aa)^{*}ba(aa)^{*} \]
%Τέλος υπάρχει μία περίπτωση που δεν καλύπτεται από τα
%προηγούμενα και η οποία είναι:
%\[w_{1_d} = ababa\]
%Εφαρμόζοντας τα αντίστοιχα όπως και στις προηγούμενες
%περιπτώσεις έχουμε:
%\[w_{2_d} = a(aa)^{*}ba(aa)^{*}ba(aa)^{*}\]
%Η τελική λύση είναι η ένωση όλων αυτών των κανονικών εκφράσεων:
%{\centering\fcolorbox{black}{lime}{$a(aa)^{*}b(aa)^{*}b(aa)^{*}
% \cup (aa)^{*}ba(aa)^{*}b(aa)^{*} \cup
%(aa)^{*}b(aa)^{*}ba(aa)^{*} \cup
%a(aa)^{*}ba(aa)^{*}ba(aa)^{*}$}\par}
%\reducevspace\reducevspace\reducevspace\reducevspace\reducevspace\reducevspace\reducevspace\reducevspace
%\vfill %stretches vertical space pushing stuff lower
%\fancyfoot[C]{\pgfornament[width=15cm, symmetry=c, color = gray!60]{88}}
%\begin{center}
	%\pgfornament[width=15cm, symmetry=c, color = gray!60]{88}
	%\pgfornament[width=15cm, symmetry=c]{89}
	%\pgfornament[width=15cm, symmetry=c]{62}
%\end{center}

\begin{center}
	%\vspace{2em}
	\noindent\rule{\linewidth}{0.5pt}
	%\vspace{2em}
\end{center}
\clearpage