\subsection{Απάντηση Υποερωτήματος (β)}
\label{ssec:Solution_1.2}
\doublespacing

CFL
$ L_1 = \{w \in \{a, b\}^* : |w|_a = 3|w|_b + 2 \}$
άρα $\vert w \vert_{min} = 2$ και συγκεκριμένα $aa$.\\
Στην περίπτωση όμως που έχουμε έστω και ένα $b$ έχουμε:
$aaa\,aa\,b$.\\
Παρατηρούμε ότι η γλώσσα δεν ορίζει συγκεκριμένη σειρά μεταξύ $a$ και
$b$ και άρα θα πρέπει να μπορούν να παραχθούν όλοι οι συνδυασμοί και με βάση τη θέση των συμβόλων και άρα
επιπρόσθετες συμβολοσειρές: $b\,aaa\,aa,\; a\,b\,aaa\,a,\;...,\;
aa\,aaa\,b$.\\
\clearpage
Για να οδηγηθούμε σε κάποιες από αυτές το ελάχιστο $aa$ διαχωρίστηκε, με το $b$ να παρεμβάλλεται ανάμεσα.
Με βάση την ελάχιστη περίπτωση καταλαβαίνουμε ότι παρόλα αυτά, τα δύο πρόσθετα $a$ θα πρέπει να απομονωθούν σε δικό
τους κανόνα.\\
Έχουμε όμως $|w|_b \in \aleph_0$ και άρα τα ανάλογα $a$, υπονοώντας χρήση αναδρομής και συγκεκριμένα όλων των
κανόνων παραγωγής εκτός του ελάχιστου $aa$.\\
Συμψηφίζοντας μπορούμε να περάσουμε στην κατασκευή των κανόνων παραγωγής:
\reducevspace\reducevspace\reducevspace\reducevspace\reducevspace\reducevspace\reducevspace
\reducevspace\reducevspace\reducevspace\reducevspace\reducevspace\reducevspace\reducevspace
%\clearpage
%\begin{multicols}{2}
\begin{itemize}
	\itemsep0em

	\item Ελάχιστη περίπτωση:\\\reducevspace
	$S \rightarrow aa$
\reducevspace\reducevspace\reducevspace\reducevspace\reducevspace\reducevspace\reducevspace
	\item Ελάχιστα $b$:\\\reducevspace
	$S \rightarrow aaR \,|\, aRa \,|\, Raa$\\\reducevspace
	$R \rightarrow baaa \,|\, abaa \,|\, aaba \,|\, aaab \,|\, \varepsilon$\\
\reducevspace\reducevspace\reducevspace\reducevspace\reducevspace\reducevspace\reducevspace
	\item $|w| = \aleph_0$:\\\reducevspace
	$S \rightarrow aaR \,|\, aRa \,|\, Raa$\\\reducevspace
	$R \rightarrow RbRaaa \,|\, aRbRaa \,|\, aaRbR \,|\, aaaRbR \,|\, \varepsilon$\\
\reducevspace\reducevspace\reducevspace\reducevspace\reducevspace\reducevspace\reducevspace
	\item Σύμπτυξη:\\\reducevspace
	$S \rightarrow RaRaR$\\\reducevspace
	$M \rightarrow bRaaa \,|\, abRaa \,|\, aabRa \,|\, aaaRbR \,|\, \varepsilon$\\
\end{itemize}
%\end{multicols}

Για να καταλάβουμε για πιο λόγο έφυγε η αναδρομή πριν από το $b$ σε όλες τις παραλλαγές του $R$ εκτός την $aaaRbR$
και άρα να υπάρχει κάποια αιτιολόγηση θα πρέπει να δώσουμε κάποια βασικά παραδείγματα:\\

\clearpage
\begin{itemize}
	\itemsep0em

	\item $bb\,aaa\,aaa\,aa \;:\; S\rightarrow RaRaR \rightarrow (baaa)aRaR \rightarrow (bRaaa)aRaR \rightarrow
			(b(bRaaa)aaa)aRaR \rightarrow (b(b\varepsilon aaa)aaa)aRaR \rightarrow (bb\,aaa\,aaa)a\varepsilon a
			\varepsilon \rightarrow bb\,aaa\,aaa\,aa$

	\item $a\,bb\,aaa\,aaa\,a \;:\; S\rightarrow RaRaR \rightarrow \varepsilon aRa\varepsilon \rightarrow
	a(bRaaa)a \rightarrow a(b(bRaaa)aaa)a \rightarrow a(b(b\varepsilon aaa)aaa)a \rightarrow a\,bb\,aaa\,aaa\,a$

	\item $aa\,bbaaa\,aaa \;:\; S\rightarrow RaRaR \rightarrow \varepsilon a\varepsilon aR \rightarrow
	aa(bRaaa) \rightarrow aa(b(bRaaa)aaa) \rightarrow aa(b(b\varepsilon aaa)aaa) \rightarrow aa\,bb\,aaa\,aaa$

	\item $...$

	\item $aa\,aaa\,bb\,aaa \;:\; S\rightarrow RaRaR \rightarrow \varepsilon a\varepsilon aR \rightarrow aa(aaabR)
	\rightarrow aa(aaab(bRaaa)) \rightarrow aa(aaab(b\varepsilon aaa)) \rightarrow aa\,aaa\,bb\,aaa$

	\item $...$

	\item $aaa\,aaa\,bb\,aa \;:\; S\rightarrow RaRaR \rightarrow Ra\varepsilon a\varepsilon \rightarrow
	(aaab)aa \rightarrow (aaaRb)aa \rightarrow (aaa(aaaRb)b)aa \rightarrow (aaa(aaa\varepsilon b)b)aa \rightarrow
	aaa\,aaa\,bb\,aa$

	\item Χρειαζόμαστε $aaaRb$ και $aaabR$. Θυμόμαστε ότι $R$ είναι αναδρομή στον ίδιο κανόνα. Παρατηρούμε ότι και
	στα δύο, είναι ίδια συμβολοσειρά, με αναδρομή πριν ή μετά το $b$ και αντικαθιστούμε με $aaaRbR$.

	\item Για να σιγουρευτούμε, ότι δεν χρειάζονται κανόνες τύπου $"...RbR..."$, έχουμε αναλύσει όλες τις
	παραλλαγές για ορθές συμβολοσειρές με έως τρία $b$, καθώς και κάποιες χαρακτηριστικές μη ορθές. Δεν τις
	συμπεριλαμβάνουμε για προφανείς λόγους. Αυτό δεν σημαίνει ότι δεν γίνεται να μας έχει ξεφύγει κάτι. Υπάρχει και
	καλύτερος τρόπος να δειχθεί γιατί λειτουργεί αυτή η γραμματική αλλά είναι επίσης φλύαρη. Παρακάτω θα δείξουμε
	εναλλακτική γραμματική.

	\item Τέλος αν πούμε ότι η CFG μας δεν περιέχει αριστερή αναδρομή.
\end{itemize}

\par
Γενικά θα πρέπει να έχουμε κατά νου ότι είναι πάρα πολύ χρήσιμο έως και απαραίτητο, εκτός των ελάχιστων περιπτώσεων
βάση συνόλου (πχ $\in \mathbb{N}_0$ ή και ελάχιστων περιπτώσεων για μη μηδενικό σύνολο (ουσιαστικά λαμβάνοντας
υπόψιν αποδοχή κενής συμβολοσειράς εάν επιτρέπεται και ελάχιστης μή κενής), καλό θα είναι να ελέγχουμε και για
διπλές εμφανίσεις των ελάχιστων μή κενών πχ $\{\forall w \neq \varepsilon \,|\, w_{min} = xy,\, \exists w_{double}
= xxyy \cup xyxy \cup xyyx \cup yxxy \cup yxyx \cup yyxx \}$

\par
Υπόψιν ότι εκτός κάποιον πολύ απλών περιπτώσεων, στις CFL δεν μπορούμε πάντα να είμαστε σίγουροι ότι η CFG μας
καλύπτει όλες τις πιθανές ορθές συμβολοσειρές της γλώσσας ή ότι είναι ελάχιστη.

\par
Θα δώσω επίσης άλλη μία λύση εν συντομία μία και έχει ήδη δειχθεί (απλά απλοποιήθηκε):

\begin{itemize}
	\item Η παρακάτω απλή και εύκολη γραμματική παρόλο που καλύπτει την γλώσσα, χρησιμοποιεί αριστερή αναδρομή κάτι
	που ενώ δεν αναφέρεται ως περιορισμός στην άσκηση, θα πρέπει να γνωρίζουμε ότι αναλόγως τα εργαλεία που θα
	χρησιμοποιήσουμε για την υλοποίηση, ίσως να προκύψει θέμα "συμβατότητας" (και είναι ένας λόγος που την
	αποφύγαμε, επιπροσθέτως στο ότι αυτή που προαναφέραμε αποτελεί ελαχιστοποίηση αυτής και άνευ αριστερής
	αναδρομής). Το πως καταλήξαμε σε αυτή φαίνεται στα βήματα της προηγούμενης προ σύμπτυξης:\\
	$G_{2b} = (\{a, b\},\, \{aa, ε\},\, \{S\rightarrow AaAaA,\,
	A\rightarrow Raaa \,|\, aRaa \,|\, aaRa \,|\, aaaR \,|\, \varepsilon,\,\\
	R\rightarrow AbA\},\, S)$
\end{itemize}

\hfill \break

\begin{tcolorbox}[colback=yellow!15!white, colframe=blue!50!white,
	fonttitle=\bfseries\Large, title = Γραμματική και συντακτικό δένδρο]
	Είμαστε έτοιμοι να προχωρήσουμε στην πλήρη περιγραφή της γραμματικής μας:\\
	$G_2 = (\{a, b\},\, \{aa, ε\},\, \{S\rightarrow RaRaR\\
	R\rightarrow bRaaa \,|\, abRaa \,|\, aabRa \,|\, aaaRbR \,|\, \varepsilon\},\, S)$

	Ήρθε η στιγμή να δώσουμε το συντακτικό δένδρο για συμβολοσειρά (τα κενά για ευκολότερη ανάγνωση) $w =
	aa\,b\,aaa\,b\,aaa,\,|w| = 10,\, |w|_a = 8,\,|w|_b = 2$:


	\begin{center}
		\Tree
		[.{S}
			[.{R}
				[.{$aab$} ]
				[.{R}
					[.{$\varepsilon$} ]
				]
				[.{$a$} ]
			]
			[.{$a$} ]
			[.{R}
				[.{$ab$} ]
				[.{R}
					[.{$\varepsilon$} ]
				]
				[.{$aa$} ]
			]
			[.{$a$} ]
			[.{R}
				[.{$\varepsilon$} ]
			]
		]
	\end{center}


	Που δίνει:$\qquad\qquad\quad\; aab\qquad\;\, a\;\;\; a\;\;\; ab\qquad\; aa\;\;\; a$\\
	Εναλλακτικά θα μπορούσαμε να το διαβάσουμε ως $(\varepsilon)\,a\,(ab(\varepsilon) aa)\,a\,(b(\varepsilon)aaa)$.

\end{tcolorbox}


\begin{center}
	%\vspace{2em}
	\noindent\rule{\linewidth}{0.5pt}
	%\vspace{2em}
\end{center}
\clearpage