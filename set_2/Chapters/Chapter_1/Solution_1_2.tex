\subsection{Απάντηση Υποερωτήματος (β)}
\label{ssec:Solution_1.2}
\doublespacing

Έχουμε CFL
$ L_1 = \{w \in \{a, b\}^* : |w|_a = 3 \times |w|_b + 2 \}$
άρα $\vert w \vert_{min} = 2$ και συγκεκριμένα $aa$.\\
Στην περίπτωση όμως που έχουμε έστω και ένα $b$ έχουμε:
$aaa\,aa\,b$.
Παρατηρούμε όμως ότι η γλώσσα δεν ορίζει συγκεκριμένη σειρά μεταξύ $a$ και
$b$ και άρα θα πρέπει να μπορούν να παραχθούν όλοι οι συνδυασμοί και με βάση τη θέση των συμβόλων και άρα
επιπρόσθετες συμβολοσειρές: $b\,aaa\,aa,\; a\,b\,aaa\,a,\; aa\,b\,aaa,\; aaa\,b\,aa,\;
aaa\,a\,b\,a$.\\
Για να οδηγηθούμε όμως σε κάποιες από αυτές το ελάχιστο $aa$ διαχωρίστηκε, με το $b$ να παρεμβάλλεται ανάμεσα.
Με βάση την ελάχιστη περίπτωση καταλαβαίνουμε ότι, παρόλα αυτά, τα δύο παραπάνω $a$ (από το τριπλάσιο των $b$)
μάλλον θα πρέπει να απομονωθούν από τα υπόλοιπα σύμβολα της λέξης σε δικό τους κανόνα.\\
Μπορούμε όμως να έχουμε έως $|w|_b = \aleph_0$ και αντίστοιχα τα $a$ που αναλογούν σε αυτά για αν
είναι σωστή η συμβολοσειρά. Αυτό, εφόσον είναι εφικτό, υπονοεί χρήση αναδρομής και συγκεκριμένα όλων των κανόνων
παραγωγής εκτός του ελάχιστου $aa$.\\
Συμψηφίζοντας τώρα αυτές τις παρατηρήσεις μπορούμε να περάσουμε στην κατασκευή των κανόνων παραγωγής. Θα το
προσεγγίσουμε σταδιακά:
%\clearpage
%\begin{multicols}{2}
\begin{itemize}
	\itemsep0em

	\item Ελάχιστη περίπτωση:\\
	$S \rightarrow aa$

	\item Ελάχιστα $b$:\\
	$S \rightarrow aaR \,|\, aRa \,|\, Raa$\\
	$R \rightarrow baaa \,|\, abaa \,|\, aaab \,|\, aaab \,|\, \varepsilon$\\

	\item $|w| = \aleph_0$:\\
	$S \rightarrow aaR \,|\, aRa \,|\, Raa$\\
	$M \rightarrow RbRaaa \,|\, aRbRaa \,|\, aaRbR \,|\, aaaRbR \,|\, \varepsilon$\\

	\item Σύμπτυξη:\\
	$S \rightarrow RaRaR$\\
	$M \rightarrow RbRaaa \,|\, aRbRaa \,|\, aaRbR \,|\, aaaRbR \,|\, \varepsilon$\\
\end{itemize}
%\end{multicols}

\begin{tcolorbox}[colback=yellow!15!white, colframe=blue!50!white,
	fonttitle=\bfseries\Large, title = Γραμματική και συντακτικό δένδρο]
	Είμαστε έτοιμοι να προχωρήσουμε στην πλήρη περιγραφή της γραμματικής μας:\\

	$G_2 = (\{a, b\},\, \{aa, ε\},\, \{S\rightarrow aa\,|\,aMa\,|\,MaaM\\
	Μ\rightarrow A_1 bM A_2 \,|\, A_2 bM A_1 \,|\, A_1 A_2 bM \,|\, Mb A_1 A_2 \,|\, ε,\;
	A_1 \rightarrow a,\; A_2 \rightarrow aa\},\, S)$

	Ήρθε η στιγμή να δώσουμε το συντακτικό δένδρο για συμβολοσειρά (τα κενά για ευκολότερη ανάγνωση) $w =
	aa\,b\,aaa\,b\,aaa,\,|w| = 10,\, |w|_a = 8,\,|w|_b = 2$:


	\begin{center}
		\Tree
		[.S
			[.$M$
				[.$A_2$
					[.$aa$ ]
				]
				[.$b$ ]
				[.$M$
					[.$ε$ ]
				]
				[.$A_1$
					[.$a$ ]
				]
			]
			[.$aa$ ]
			[.$M$
				[.$b$ ]
				[.$M$
					[.$ε$ ]
				]
				[.$A_1$
					[.$a$ ]
				]
				[.$A_2$
					[.$aa$ ]
				]
			]
		]
	\end{center}


	Που δίνει:$\qquad\qquad\;\;\, aa\quad b\qquad\;\;\; a\quad\,  aa\;\;\; b\qquad\;\;\; a\quad\, aa$

\end{tcolorbox}