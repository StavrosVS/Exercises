\subsection{Απάντηση Υποερωτήματος (β)}
\label{ssec:Solution_3.2}
\doublespacing

Η άσκηση μας ζητά να αποδείξουμε εάν η παρακάτω γλώσσα είναι CF ή όχι:
\[L = \{a^mb^nc^k\,:\,n,\,k,\,m\in\mathbb{N}_0,\, n=3m+2k\}\]

\par
Υπάρχουν πάρα πολλοί τρόποι να το κάνουμε αυτό. Μεταξύ των οποίων είναι η απόπειρα κατασκευής CFG της γλώσσας, όπου
αν τα καταφέρουμε αυτόματα αποδεικνύουμε ότι είναι CFL, ενώ μία άλλη μέθοδος είναι μέσω λήμματος άντλησης. Υπάρχουν
αρκετοί ακόμη μέθοδοι αλλά δεν θα τις αναφέρουμε ούτε θα τις δείξουμε όλες.

\begin{tcolorbox}[colback=yellow!15!white, colframe=blue!50!white,
	fonttitle=\bfseries\Large, title = Απόδειξη με απόπειρα κατασκευής CFG]
	\begin{itemize}
		\itemsep1em

		\item Έχουμε συγκεκριμένη θέση έκαστου συμβόλου δηλαδή πρώτα όλα τα $a$, έπειτα όλα τα $b$ και τελευταία
		όλα τα $c$.

		\item Συγκεκριμένη η θέσης και ουσιαστικά το κάθε σύμβολο έχει συγκεκριμένη προτεραιότητα (δεν
		αναμιγνύονται πχ δεν υπάρχει bbabbcb παρόλο που η αναλογία συμβόλων είναι ορθή). Ακριανά
		σύμβολα ($a,\,c$)καθορίζουν το πλήθος των μεσαίων ($b$). Άρα μπορούμε να σπάσουμε όλες τις συμβολοσειρές
		της γλώσσας σε δύο μέρη: ένα αριστερό ($L_{LHS} = \{w = \{a^m\, b^{3m}\}^* \,:\, m \in \mathbb{N}_0\}
		$) και ένα δεξιό ($L_{RHS} = \{ w = \{b^{2k}\, c^k\}^* \,:\, k \in \mathbb{N}_0 \}$). Η λύση θα είναι
		η σύνθεση αυτών τον δύο και συγκεκριμένα:\\ $\{w \in L,\, w_1 \in L_{LHS},\, w_2 \in L_{RHS} \,\vert\, w =
		w_1
		w_2\}$

		\item Για $L_{LHS} \;:\; S \rightarrow \varepsilon \,|\, A,\quad A \rightarrow \varepsilon \,|\, aAbbb $

		\item Για $L_{RHS} \;:\; S \rightarrow \varepsilon \,|\, C,\quad C \rightarrow \varepsilon \,|\, bbCc $

		\item Άρα $\,:\, S \rightarrow \varepsilon \,|\, AC,\quad A \rightarrow \varepsilon \,|\, aAbbb,\quad C
		\rightarrow \varepsilon \,|\, bbCc$

		\item $G = (\{a,\,b,\,c\},\quad \{\varepsilon\},$\\
		$\{S \rightarrow \varepsilon \,|\, AC,\; A \rightarrow \varepsilon \,|\, aAbbb,\; C \rightarrow \varepsilon
		\,|\, bbCc\},\quad S)$


	\end{itemize}
\end{tcolorbox}


\begin{center}
	%\vspace{2em}
	\noindent\rule{\linewidth}{0.5pt}
	%\vspace{2em}
\end{center}
\clearpage
