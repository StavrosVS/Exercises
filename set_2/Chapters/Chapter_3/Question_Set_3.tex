\section{Άσκηση 3η - Mη ντετερμινισμός και κανονικότητα αυτομάτων:}
\label{sec:Exercise_3}
\doublespacing

[25\%] Έστω το μη ντετερμινιστικό αυτόματο $M$ που εικονίζεται παρακάτω:

\bm{\textcolor{blue}{(α)}} Κατασκευάστε αναλυτικά ένα ισοδύναμο ντετερμινιστικό αυτόματο $M'$

\bm{\textcolor{blue}{(β)}} Κατασκευάστε αναλυτικά την κανονική έκφραση για την $L(Μ)$ με σειρά απαλοιφής
$q3,\,q1,\,q5,\,q2.$

\hfill \break
\begin{figure}[!htb] \centering
	\begin{tikzpicture} [blue, node distance = 2.5cm, on grid, font=\sffamily\large\bfseries]
		% Help grid
		%	\draw [help lines] (-1,4) grid (8,-10);
		% Start Node : Every other node is measured based to this one
		\node (q2) [state, black, scale = 1.3, initial below, initial distance = 4mm, initial text = {Αρχή},
		top color = green!80, bottom color = green!20] {$q_{2}$};

		% Middle Nodes
		\node (q1) [state, black, scale = 1.3, top color = gray!80, bottom color = gray!20, left = of q2] {$q_{1}$};
		\node (q5) [state, black, scale = 1.3, top color = gray!80, bottom color = gray!20, right = of q2]
		{$q_{5}$};
		\node (q3) [state, black, scale = 1.3, top color = gray!80, bottom color = gray!20, right = of q5]
		{$q_{3}$};

		% Final Node
		\node (q4) [state, black, scale = 1.3, accepting, top color = red!80, bottom color = red!20, , double
		distance = 3pt, double = red!20, right = of q3] {$q_{4}$};

		% Draw Arrows/Connections
		\path [-stealth, thick]
		(q2) edge [bend left = 20] node [above] {$a,\,b$}(q5)
		(q1) edge [] node [above] {$\mathcal{ε}$} (q2)
		(q5) edge [] node [above] {$a$} (q3)
		(q5) edge [bend right = 60] node [below] {$a$} (q1)
		(q5) edge [bend left = 20] node [below] {$b$} (q2)
		(q3) edge [bend left = 60] node [above] {$\mathcal{ε}$} (q2)
		(q3) edge [] node [above] {$b$} (q4);

	\end{tikzpicture}
	%\end{flushleft}
	\captionsetup{labelformat=empty}
	\caption{Αρχικό DFA}
	%\label{fig:sub1}
\end{figure}

\clearpage