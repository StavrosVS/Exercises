\subsection{Απάντηση Υποερωτήματος (α)}
\label{ssec:Solution_3.1}
\doublespacing

\begin{itemize}
	\itemsep0em

	\item $\mathcal{P} := \{\forall\,L_{DCF}\in\mathcal L_{\mathrm{DCF}},\;\forall\,L_{reg}\in\mathcal
	L_{\mathrm{REG}}:
	\;L_{DCF}\cap L_{reg}\;\in\;\mathcal L_{\mathrm{CF}}\}$


	\item $\mathcal{L}_{fin} \subset \mathcal{L}_{reg} \subset \mathcal{L}_{DCF} \subset \mathcal{L}_{NCF} =
	\mathcal{L}_{CF}$

	\item

	\item
\end{itemize}
\par
Κατόπιν βρίσκουμε $q_{\text{αρχικό}}$ του $ε$-NFA και συνεχίζουμε με όλα τα E(q):
\hfill \break
\setlength{\arrayrulewidth}{0.5mm}
\setlength{\tabcolsep}{18pt}
\renewcommand{\arraystretch}{1}

\[
\mathcal L_{\mathrm{DCFL}}
\;\cap\;
\mathcal L_{\mathrm{REG}}
\;\subseteq\;
\mathcal L_{\mathrm{DCFL}}
\;\subseteq\;
\mathcal L_{\mathrm{CFL}}.
\]


\par
Βρήκαμε τον αρχικό, τελικό και ενδιάμεσους κόμβους καθώς και τις μεταβάσεις τους και άρα είμαστε έτοιμοι να
προχωρήσουμε στην κατασκευή του DFA:

\begin{tcolorbox}[colback=yellow!15!white, colframe=blue!50!white,
	fonttitle=\bfseries\Large, title = Τελικό DFA $M'$]
	\centering
	\begin{tikzpicture} [blue, node distance = 3.5cm, on grid, font=\sffamily\large\bfseries]
		% Help grid
		%	\draw [help lines] (-1,4) grid (8,-10);
		% Start Node : Every other node is measured based to this one
		\node (q0) [state, black, scale = 1.3, initial left, initial distance = 4mm, initial text = {Αρχή},
		top color = green!80, bottom color = green!20] {$q_{0}$};

		% Middle Nodes
		\node (q2) [state, black, scale = 1.3, top color = gray!80, bottom color = gray!20, below = of q0]
		{$q_{2}$};
		\node (q1) [state, black, scale = 1.3, top color = gray!80, bottom color = gray!20, right = of q2]
		{$q_{1}$};

		% Final Node
		\node (q3) [state, black, scale = 1.3, accepting, top color = red!80, bottom color = red!20, , double
		distance = 3pt, double = red!20, left = of q2] {$q_{3}$};

		% Draw Arrows/Connections
		\path [-stealth, thick]
		(q0) edge [bend left = 10] node [right] {$a,\,b$}(q1)
		(q1) edge [bend left = 10] node [below] {$a$} (q2)
		(q1) edge [bend left = 10] node [left] {$b$} (q0)
		(q2) edge [bend left = 10] node [above] {$a$} (q1)
		(q2) edge [bend left = 10] node [below] {$b$} (q3)
		(q3) edge [bend left = 10] node [above] {$a$} (q2)
		(q3) edge [] node [above] {$b$} (q0);

	\end{tikzpicture}
\end{tcolorbox}

\begin{center}
	%\vspace{2em}
	\noindent\rule{\linewidth}{0.5pt}
	%\vspace{2em}
\end{center}
\clearpage