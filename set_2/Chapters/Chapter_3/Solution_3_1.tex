\subsection{Απάντηση Υποερωτήματος (α)}
\label{ssec:Solution_3.1}
\doublespacing
Η πρόταση "η τομή μίας ντετερμινιστικής γλώσσας χωρίς συμφραζόμενα με μία πεπερασμένη γλώσσα είναι πάντα γλώσσα
χωρίς συμφραζόμενα" είναι αληθής όπως θα αποδείξουμε παρακάτω:


\reducevspace\reducevspace\reducevspace


\begin{tcolorbox}[colback=yellow!15!white, colframe=blue!50!white,
	fonttitle=\bfseries\Large, title = Απόδειξη]
	\centering
\begin{itemize}
	\itemsep0em

	\item $\mathcal{P} := \{\forall\,L_{DCF}\in\mathcal L_{\mathrm{DCF}},\;\forall\,L_{REG}\in\mathcal
	L_{\mathrm{REG}}:\;L_{DCF}\cap L_{REG}\;\in\;\mathcal L_{\mathrm{CF}}\}$
	\reducevspace\reducevspace\reducevspace\reducevspace\reducevspace\reducevspace
	\reducevspace\reducevspace\reducevspace\reducevspace\reducevspace\reducevspace
	\reducevspace\reducevspace\reducevspace\reducevspace\reducevspace\reducevspace
	\reducevspace\reducevspace\reducevspace\reducevspace\reducevspace\reducevspace
	\begin{flushright}\hypertarget{3.1.1}{\bf{(1)}}\end{flushright}


	\item $\mathcal{L}_{FIN} \subset \mathcal{L}_{REG} \subset \mathcal{L}_{DCF} \subset \mathcal{L}_{NCF} =
	\mathcal{L}_{CF}$
	\reducevspace\reducevspace\reducevspace\reducevspace\reducevspace\reducevspace
	\reducevspace\reducevspace\reducevspace\reducevspace\reducevspace\reducevspace
	\reducevspace\reducevspace\reducevspace\reducevspace\reducevspace\reducevspace
	\reducevspace\reducevspace\reducevspace\reducevspace\reducevspace\reducevspace
	\begin{flushright}\hypertarget{3.1.2}{\bf{(2)}}\end{flushright}

	\item $\forall (A,\, B) \ni A \subset B \rightarrow A \cap B = A,\, Α\subset B$
	\reducevspace\reducevspace\reducevspace\reducevspace\reducevspace\reducevspace
	\reducevspace\reducevspace\reducevspace\reducevspace\reducevspace\reducevspace
	\reducevspace\reducevspace\reducevspace\reducevspace\reducevspace\reducevspace
	\reducevspace\reducevspace\reducevspace\reducevspace\reducevspace\reducevspace
	\begin{flushright}\hypertarget{3.1.3}{\bf{(3)}}\end{flushright}

	\item $A \subset B \subset...\subset N \Rightarrow A \subset N$
	\reducevspace\reducevspace\reducevspace\reducevspace\reducevspace\reducevspace
	\reducevspace\reducevspace\reducevspace\reducevspace\reducevspace\reducevspace
	\reducevspace\reducevspace\reducevspace\reducevspace\reducevspace\reducevspace
	\reducevspace\reducevspace\reducevspace\reducevspace\reducevspace\reducevspace
	\begin{flushright}\hypertarget{3.1.4}{\bf{(4)}}\end{flushright}

	\item $\overset{\hyperlink{3.1.2}{(2)} \hyperlink{3.1.4}{(4)}}{\Longrightarrow} \mathcal{L}_{FIN} \subset
	\mathcal{L}_{DCF}$
	\reducevspace\reducevspace\reducevspace\reducevspace\reducevspace\reducevspace
	\reducevspace\reducevspace\reducevspace\reducevspace\reducevspace\reducevspace
	\reducevspace\reducevspace\reducevspace\reducevspace\reducevspace\reducevspace
	\reducevspace\reducevspace\reducevspace\reducevspace\reducevspace\reducevspace
	\begin{flushright}\hypertarget{3.1.5}{\bf{(5)}}\end{flushright}

	\item $\mathcal{L}_{FIN} \cap \mathcal{L}_{DCF}\!\!\overset{\hyperlink{3.1.3}{(3)}\hyperlink{3.1.5}{(5)}}{=}\!\!
	\mathcal{L}_{FIN} \subset \mathcal{L}_{DCF} \subset \mathcal{L}_{CF}
	\overset{\hyperlink{3.1.4}{(4)}}{\Rightarrow} \mathcal{L}_{FIN} \subset \mathcal{L}_{CF} \Rightarrow
	\mathcal{L}_{FIN} \in \mathcal{L}_{CF}$
	\reducevspace\reducevspace\reducevspace\reducevspace\reducevspace\reducevspace
	\reducevspace\reducevspace\reducevspace\reducevspace\reducevspace\reducevspace
	\reducevspace\reducevspace\reducevspace\reducevspace\reducevspace\reducevspace
	\reducevspace\reducevspace\reducevspace\reducevspace\reducevspace\reducevspace
	\begin{flushright}\hypertarget{3.1.6}{\bf{(6)}}\end{flushright}

	\item $\mathcal{Q} \overset{\hyperlink{3.1.6}{(6)}}{:=} \{\forall\,L_{DCF}\in\mathcal
	L_{\mathrm{DCF}},\;\forall\,L_{REG}\in\mathcal
	L_{\mathrm{REG}}:\;L_{DCF}\cap L_{REG}\;\in\;\mathcal L_{\mathrm{CF}}\}$
	\reducevspace\reducevspace\reducevspace\reducevspace\reducevspace\reducevspace
	\reducevspace\reducevspace\reducevspace\reducevspace\reducevspace\reducevspace
	\reducevspace\reducevspace\reducevspace\reducevspace\reducevspace\reducevspace
	\reducevspace\reducevspace\reducevspace\reducevspace\reducevspace\reducevspace
	\begin{flushright}\hypertarget{3.1.7}{\bf{(7)}}\end{flushright}

	\item $\mathcal{P}^{\hyperlink{3.1.1}{(1)}} = \mathcal{Q}^{\hyperlink{3.1.7}{(7)}} = $ True $ \Rightarrow
	\mathcal{P} = $ True

\end{itemize}
\end{tcolorbox}

\clearpage
Για να εξηγήσω την λογική, δείξαμε ότι:
\begin{enumerate}
	\item Tομή μεταξύ ενός συνόλου με ενός υποσυνόλου (γνησίου ή μη) παράγει μόνο το ίδιο το υποσύνολο.
	Αποδεικνύεται μέσω της απόδειξης για του νόμου της απορρόφησης μεταξύ υποσυνόλου και συνόλου. Θα παραθέσω τις
	όποιες αποδείξεις παρακάτω.

	\item Tο χρησιμοποιήσαμε για να δείξουμε ότι τομή πεπερασμένων γλωσσών και ντετερμινιστικών χωρίς
	συμφραζόμενα γλωσσών, παράγει αποκλειστικά πεπερασμένες γλώσσες. Ο λόγος είναι ότι οι πεπερασμένες είναι
	υποσύνολο των DCFL.

	\item Κατόπιν δείξαμε ότι λόγο του ότι αν έχουμε μία αλυσίδα συνόλων με κάθε επόμενο να είναι
	υπερσύνολο του κάθε προηγούμενου (έστω και γνήσιο) τότε μπορούμε να πούμε ότι οποιοδήποτε μικρότερο
	σύνολο είναι εντός οποιουδήποτε μεγαλύτερου. Αποδεικνύεται, διαισθητικά, μέσω του ιδίου παίρνοντας οποιοδήποτε
	στοιχείο από το οποιοδήποτε μικρότερο σύνολο και δείχνοντας ότι ανήκει σε όλα τα υπερσύνολα.

	\item Τα συνδυάσαμε συλλογιζόμενοι ότι τομή πεπερασμένης με DCFL παράγει πεπερασμένη και αφού η
	πεπερασμένη είναι υποσύνολο κανονικής, που είναι υποσύνολο DCFL, που είναι υποσύνολο CFL, τότε η
	πεπερασμένη γλώσσα που παίρνουμε ως {αποτέλεσμα τομής πεπερασμένης με DCFL}, είναι επίσης υποσύνολο CFL γλωσσών
	και άρα κάθε στοιχείο της είναι και στοιχείο των CFL γλωσσών.

	\item Άρα η τομή πεπερασμένης γλώσσας και ντετερμινιστικής γλώσσας χωρίς συμφραζόμενα παράγει πάντα πεπερασμένη
	γλώσσα και αφού οι πεπερασμένες γλώσσες είναι υποσύνολα των γλωσσών χωρίς συμφραζόμενα, τότε οποιαδήποτε
	πεπερασμένη γλώσσα, συμπεριλαμβανομένης και αυτής που παράγει τομή πεπερασμένης με DCFL, ανήκει στις γλώσσες
	χωρίς συμφραζόμενα.

\end{enumerate}

\clearpage

Απόδειξη $A \subset B \rightarrow A \cap B \Rightarrow A$ και $A \subset B \subset C \Rightarrow A \subset C$:
\begin{enumerate}
	\item Ορισμός γνήσιου υποσυνόλου:\\
	$Α\subset B \overset{def}{=}  \bigl[\forall x(x\in A \Rightarrow x\in B)\bigr]\land \bigl[\exists y(y\in B
	\land y\notin A)\bigr]$

	 \item Απόδειξη $A \subset B \subset C \Rightarrow A\subset C$:\\
	 $B\subset C = \forall x(x\in B \Rightarrow x\in C)$,\qquad
	 $A\subset B = \forall x(x\in A \Rightarrow x\in B)$\\
	 $A \subset B \subset C = \bigl[\forall x(x\in A \Rightarrow x\in B \Rightarrow x\in C)\bigr]\land
	  \bigl[\exists y(y\in C \land y\notin B \Rightarrow y\notin A)\bigr]\Rightarrow$\\
	  $\Rightarrow [\forall x(x\in A \Rightarrow x\in C)]\land [\exists y(y\in C \land y\notin A)] = A\subset C$

	 \item Ορισμός τομής:\\
	 $A\cap B \overset{def}{=}\{x\,|\,x\in A, x\in B\}$

	 \item Απόδειξη $A\subset B \rightarrow A\cap B = A$:\\
	 $A\cap B = C \rightarrow \forall x\bigl[(x\in A \land x\in B)\Leftrightarrow x\in C\bigr]$ αλλά $A\subset B
	 \rightarrow \forall x(x\in A \Rightarrow x\in B)$\\
	 οπότε στη συγκεκριμένη περίπτωση (και όλες τις αντίστοιχες) έχουμε:\\
	 $A\cap B = C \rightarrow \bigl[\forall x(x\in A \Rightarrow x\in B \Rightarrow x\in C)\bigr]\land
	 \bigl[\exists y(y\notin A \land y\in B \Rightarrow y\notin C)\bigr] \Rightarrow\\
	 \bigl[\forall x(x\in A \Rightarrow x\in C)\bigr] \land \bigl[\forall y(y\notin A \Rightarrow y\notin C)\bigr]
	 \Rightarrow A = C
	 \Rightarrow A\cap B = A$

\end{enumerate}


\begin{center}
	%\vspace{2em}
	\noindent\rule{\linewidth}{0.5pt}
	%\vspace{2em}
\end{center}
\clearpage